\pdfminorversion=4
\documentclass[]{article}
\usepackage[utf8]{inputenc}
\usepackage{amssymb,latexsym,amsmath}
\usepackage[a4paper,top=3cm,bottom=2cm,left=3cm,right=3cm,marginparwidth=1.75cm]{geometry}
\usepackage{graphicx}
\usepackage[colorlinks=true, allcolors=blue]{hyperref}
\begin{document}

Il progetto \textbf{STAR} (Sistema di Tracciamento dello Spazio Aereo e Rotte) è una banca dati pensata per gestire in modo integrato e monitorare il traffico aereo sia civile che militare. Questo sistema segue tutte le fasi operative di un volo, dalla pianificazione della rotta e prenotazione degli slot negli aeroporti, fino alla storicizzazione delle operazioni effettuate e alla gestione dei dati telemetrici


La banca dati si concentra su tre aree principali:

\begin{itemize}
    \item \textbf{Infrastruttura e Logistica}: Qui si gestiscono gli aeroporti, sia hub che regionali e militari, i gate e tutta la complessa logistica di carico merce (container) e passeggeri.
    
    \item \textbf{Flotta e Risorse Umane}: Questo aspetto si occupa in dettaglio degli aeromobili, con tutte le loro configurazioni, e del personale specializzato come piloti, controllori e manutentori, tenendo traccia delle loro abilitazioni, turni e certificazioni.

    
    \item \textbf{Sicurezza e Monitoraggio}: Archiviazione dei log operativi (rilevazioni GPS campionate, parametri tecnici e messaggi di errore) finalizzata alla manutenzione predittiva e alla gestione dei protocolli di sicurezza per le aree militari.
\end{itemize}

L'obiettivo è quello di supportare le decisioni per ottimizzare le risorse aeroportuali e garantire una completa tracciabilità degli eventi di volo e di manutenzione.


\end{document}