\pdfminorversion=4
\documentclass[]{article}
\usepackage[utf8]{inputenc}
\usepackage{amssymb,latexsym,amsmath}
\usepackage[a4paper,top=3cm,bottom=2cm,left=2cm,right=3cm,marginparwidth=1.75cm]{geometry}
\usepackage{graphicx}
\usepackage[colorlinks=true, allcolors=blue]{hyperref}
\begin{document}

\section{Progettazione Concettuale}

La Figura 1 presenta il diagramma Entità-Relazione (E-R) che riassume i concetti emersi dall'analisi dei requisiti e le loro connessioni. Il sistema rappresenta una rete aeroportuale suddivisa in tre categorie di aereoporti: Hub Internazionali, Aeroporti Regionali e Basi Militari. Questa classificazione forma una gerarchia totale e disgiunta, quindi ogni aeroporto deve appartenere, in modo esclusivo, a una di queste categorie. Gli Hub Internazionali comprendono anche diversi Gate, che sono modellati come entità deboli, identificati dalla coppia (CodAeroporto, NumeroGate). Un altro elemento chiave del dominio riguarda gli Aerei, le cui specifiche tecniche derivano dall'entità Tipo Aereo, che include il catalogo dei modelli registrati nel sistema. Gli aerei sono organizzati tramite una generalizzazione totale e sovrapposta, differenziando tra Aerei di Linea, Aerei Cargo, Aerei Privati e Aerei Militari. Questa scelta di modellazione riflette che un singolo velivolo può operare in più ruoli contemporaneamente, come nel caso di configurazioni combinate passeggeri-cargo o di freighter utilizzati in contesti militari. Gli Aerei di Linea hanno anche una Configurazione interna, definita rispetto all'entità Classe Passeggeri, che rappresenta le varie categorie di servizio disponibili a bordo (per esempio, Economy o Business). A ciascuna classe possono essere associati uno o più Servizi Aggiuntivi, modellati come entità deboli identificate dalla coppia (NomeServizio, ID Classe). La gestione operativa dei voli è affidata all'entità Volo, che rappresenta l'evento centrale del sistema. Le relazioni di Partenza e Arrivo collegano ogni volo rispettivamente all'aeroporto di partenza e quella di destinazione. Ogni volo è effettuato da un singolo aereo tramite la relazione Esegue ed è monitorato da uno o più Controllori di Volo attraverso la relazione Monitoraggio. L'entità Personale descrive le risorse umane coinvolte nelle operazioni aeroportuali ed è organizzata in una generalizzazione parziale e disgiunta nelle sottoclassi Pilota, Controllore di Volo e Manutentore. Durante l'esecuzione di un volo, vengono registrati i Log di Transito, modellati come entità deboli identificate dalla coppia (IDVolo, Timestamp). La gerarchia è totale e separata e include tre tipi di log: Log Sensore GPS, Log Dati di Volo e Log Errori. I log d'errore possono generare notifiche per i manutentori tramite la relazione Gestione Alert. Per quanto riguarda la gestione delle merci, l'entità Container rappresenta le unità di carico trasportate nel sistema. La relazione Imbarco Merce associa ogni container ai voli sui quali viene caricato, registrando peso e contenuto dell'operazione. La relazione Trasporto consente di ricostruire la cronologia dei container movimentati da ciascun aereo cargo, indipendentemente dai voli. L'entità Rotta Pianificata descrive i collegamenti standard tra aeroporti e viene associata ai voli tramite la relazione Assegnazione Rotta. La Tabella 1, che trovi nella sezione successiva, riassume tutte le entità e relazioni illustrate nello schema E-R, indicando attributi e identificatori.  

\subsection*{Vincoli non rappresentabili nello schema E-R}

Alcune regole del dominio non possono essere rappresentate direttamente nel diagramma
E--R, ma costituiscono comunque vincoli fondamentali del sistema. Tra i principali:

\begin{itemize}

    \item \textbf{Un aereo classificato come Militare non può essere posseduto da una compagnia aerea commerciale}.  
    Lo schema consente la relazione \emph{Possesso} con qualsiasi aereo, ma il dominio richiede il vincolo:
    \[
        a \in \text{Militare} \;\Rightarrow\; \neg \exists c \; \text{Possesso}(c,a).
    \]

    \item \textbf{Un container può essere imbarcato su un volo solo se il peso registrato non supera la capacità di carico dell’aereo che esegue quel volo}.  
    Questo vincolo incrocia attributi e relazioni diverse (\emph{ImbarcoMerce}, \emph{Esegue}, \emph{AereoCargo}) ed è quindi non rappresentabile graficamente:
    \[
        \text{ImbarcoMerce}(c,v) \;\Rightarrow\; 
        \text{PesoRegistrato}(c,v) \leq \text{VolumeCaricoMax}(\text{AereoEsegue}(v)).
    \]

    \item \textbf{Un controllore di volo può monitorare un volo solo se è assegnato a un aeroporto coerente con quel volo}.  
    In particolare, se un controllore monitora un volo, deve presidiare almeno uno degli aeroporti di partenza o arrivo:
    \[
        \text{Monitoraggio}(p,v) \;\Rightarrow\; 
        \exists a \big( \text{Presidio}(p,a) 
        \wedge (a = \text{Partenza}(v) \;\vee\; a = \text{Arrivo}(v)) \big).
    \]

    \item \textbf{La gestione degli errori dipende dallo stato del ticket}.  
    In particolare, uno stato diverso da ``non assegnato'' richiede che esista almeno un manutentore associato tramite \emph{Gestione~Alert}.  
    Il legame tra stato del log e presenza di un manutentore non è esprimibile nel solo schema E--R.

    \item \textbf{Un membro dell’equipaggio può operare come Pilota solo se è abilitato al modello dell’aereo che esegue il volo}.  
    Questo vincolo incrocia \emph{Equipaggio}, \emph{Pilota}, \emph{AbilitazioneVolo}, \emph{Esegue} e \emph{TipoAereo}:
    \[
        \text{Equipaggio}(p,v) \wedge \text{Ruolo}(p,v)=\text{``Pilota''}
        \;\Rightarrow\;
        \exists t\big(
            \text{AbilitazioneVolo}(p,t)
            \wedge
            t = \text{TipoAereo}(\text{AereoEsegue}(v))
        \big).
    \]

\end{itemize}

\begin{table}[h!]
\centering
\begin{tabular}{|p{4cm}|p{4cm}|p{4cm}|p{4cm}|}
\hline
\textbf{Entità} & \textbf{Descrizione} & \textbf{Attributi} & \textbf{Identificatore} \\
\hline

Aeroporto & Struttura aeroportuale & codice, nome, città, nazione, coordinate, numero\_piste, numero\_voli\_giornalieri& codice\\
\hline

HubInternazionale & Tipo di aeroporto & numero\_terminal & \\
\hline

AeroportoRegionale & Tipo di aeroporto & lunghezza\_pista\_max, notturno& \\
\hline

BaseMilitare & Aeroporto militare & codice\_nato, livello\_sicurezza& \\
\hline

Gate& Punto di imbarco & numero, piano, tunnel\_imbarco& (codice, numero)\\
\hline

ProtocolloSicurezza & Procedure militari & id, nome, descrizione& id\\
\hline

Aereo & Velivolo & matricola, data\_immatricolazione, ore\_volo\_accumulate& matricola\\
\hline

AereoDiLinea & Sottotipo aereo & wifi, intrattenimento& \\
\hline

AereoCargo & Sottotipo aereo & volume\_carico\_max& \\
\hline

AereoPrivato & Sottotipo aereo & nome\_proprietario, servizio& \\
\hline

AereoMilitare & Sottotipo aereo & codice\_missione, ente\_operativo& \\
\hline

TipoAereo & Catalogo modelli & codice\_modello, costruttore, autonomia, velocita\_crociera& codice\_modello\\
\hline

Componente& Componenti tecnici standard & numero, descrizione, modello\_aereo& (codice\_modello, numero)\\
\hline

Personale & Risorse umane & id, nome, cognome, data\_nascita, data\_assunzione& id\\
\hline

Pilota & Sottoclasse personale & numero\_licenza, ore\_volo\_totali, ultima\_visita\_medica& \\
\hline

ControlloreVolo & Sottoclasse personale & livello\_certificazione, scadenza\_abilitazione& \\
\hline

Manutentore & Sottoclasse personale & qualifica& \\
\hline

Volo & Evento operativo & id\_volo, codice\_volo, data\_programmata, data\_partenza\_effettiva, stato& id\_volo\\
\hline

LogTransito& Rilevazioni telemetriche & timestamp, altitudine, velocità& (id\_volo, timestamp)\\
\hline

LogGPS & Sottotipo log & latitudine, longitudine& \\
\hline

LogDatiVolo & Sottotipo log & livello\_carburante, temperatura\_esterna, pressione\_cabina& \\
\hline

LogErrori & Sottotipo log & codice\_errore, messaggio, livello\_severità& \\
\hline

FasciaOraria & Slot aeroportuali & id\_slot, orario\_inizio, orario\_fine, tipo& id\_slot\\
\hline

RottaPianificata & Percorso standard & id\_rotta, nome\_rotta, consumo\_carburante, distanza, durata\_stimata& id\_rotta\\
\hline

Container & Unità di carico & codice\_container, peso\_tara, tipologia& codice\_container\\
\hline

ClassePasseggeri & Categoria servizio & id\_classe, spazio\_gambe& id\_classe\\
\hline

ServizioAggiuntivo& Servizi associati alle classi & nome\_servizio, descrizione, classe\_appartenenza& (nome\_servizio, id\_classe)\\
\hline

CompagniaAerea & Operatore aeroportuale & codice, nome, nazione, sito\_web& codice\\
\hline

\end{tabular}
\caption{Entità individuate dall’analisi dei requisiti, con attributi e identificatori.}
\end{table}

\begin{table}[h!]
\centering
\begin{tabular}{|p{4cm}|p{4cm}|p{4cm}|p{4cm}|}
\hline
\textbf{Relazione} & \textbf{Descrizione} & \textbf{Componente} & \textbf{Attributi} \\
\hline

Possiede & Un Hub Internazionale possiede uno o più gate & Hub internazionale, Gate & -- \\
\hline

Applica & Una base militare applica protocolli di sicurezza & BaseMilitare, ProtocolloSicurezza & data\_inizio\_validità \\
\hline

DivietoOperativo & Un aeroporto regionale vieta certi tipi di aereo & Regionale, TipoAereo & -- \\
\hline

Esecuzione & Un aereo esegue un volo & Aereo, Volo & -- \\
\hline

Partenza & Un volo parte da un aeroporto & Volo, Aeroporto & -- \\
\hline

Arrivo & Un volo arriva in un aeroporto & Volo, Aeroporto & -- \\
\hline

AssegnazioneRotta & Una rotta viene associata a un volo & RottaPianificata, Volo & -- \\
\hline

PrenotazioneDecollo & Un volo prenota uno slot di decollo & Volo, FasciaOraria & -- \\
\hline

PrenotazioneAtterraggio & Un volo prenota uno slot di atterraggio & Volo, FasciaOraria & -- \\
\hline

Equipaggio & Piloti e assistenti operano su un volo & Personale, Volo & ruolo, ore\_servizio \\
\hline

Monitoraggio & Un controllore supervisiona un volo & ControlloreVolo, Volo & fase\_volo \\
\hline

Presidio & Un controllore è assegnato a un aeroporto & ControlloreVolo, Aeroporto & settore, data\_fine, data\_inizio \\
\hline

Traccia & Un volo genera log di transito & Volo, LogTransito & -- \\
\hline

GestioneAlert & Un manutentore gestisce un log di errore & Manutentore, LogErrori & stato\_ticket, data\_presa\_carico \\
\hline

ImbarcoMerce & Un container è imbarcato su un volo & Container, Volo & peso\_registrato, data\_imbarco, contenuto \\
\hline

Trasporto & Uno storico dei container trasportati da un aereo cargo & AereoCargo, Container & data\_carico, posizione\_stiva \\
\hline

Configurazione & Una classe passeggeri è configurata su un aereo di linea & AereoDiLinea, ClassePasseggeri & numero\_posti \\
\hline

Dotazione & Componenti tecniche installate su un modello di aereo & TipoAereo, Componente & -- \\
\hline

AbilitazioneVolo & Un pilota è abilitato a un modello di aereo & Pilota, TipoAereo & data\_conseguimento, data\_scadenza \\
\hline

AssegnazioneGate & Una compagnia aerea ha uno o più gate assegnati & CompagniaAerea, Gate & -- \\
\hline

Possesso & Una compagnia possiede uno o più aerei & CompagniaAerea, Aereo & -- \\
\hline

CertificazioneManutenzione & Un manutentore certifica un aereo dopo un intervento & Manutentore, Aereo & data\_scadenza \\
\hline

Modello & Un aereo è associato a un modello tecnico & Aereo, TipoAereo & -- \\
\hline

Offre & Una classe passeggeri include un servizio aggiuntivo & ClassePasseggeri, ServizioAggiuntivo & -- \\
\hline

Origine & Una rotta pianificata ha un aeroporto di partenza & RottaPianificata, Aeroporto & -- \\
\hline

Destinazione & Una rotta pianificata ha un aeroporto di arrivo & RottaPianificata, Aeroporto & -- \\
\hline

Disponibilità & Una fascia oraria è disponibile in un aeroporto & FasciaOraria, Aeroporto & -- \\
\hline

\end{tabular}
\caption{Relazioni individuate nello schema concettuale, con componenti e attributi.}
\end{table}

\end{document}