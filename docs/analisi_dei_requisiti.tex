\pdfminorversion=4
\documentclass[]{article}
\usepackage[utf8]{inputenc}
\usepackage{amssymb,latexsym,amsmath}
\usepackage[a4paper,top=3cm,bottom=2cm,left=3cm,right=3cm,marginparwidth=1.75cm]{geometry}
\usepackage{graphicx}
\usepackage[colorlinks=true, allcolors=blue]{hyperref}
\begin{document}

\section{Analisi dei Requisiti}

Questa sezione riassume i requisiti informativi richiesti dal sistema da sviluppare. 
L'obiettivo è descrivere i concetti principali, le loro caratteristiche rilevanti e 
le relazioni che intercorrono tra essi.

\subsection*{Aeroporti}
Ogni aeroporto è identificato da un codice internazionale e contiene le seguenti informazioni:
\begin{itemize}
    \item codice IATA;
    \item nome;
    \item città in cui si trova;
    \item coordinate GPS;
    \item numero di piste disponibili.
    \item numero di voli giornalieri.
\end{itemize}
Gli aeroporti si suddividono in tre categorie (gerarchia totale e disgiunta):
\begin{itemize}
    \item \textbf{Hub Internazionale}, caratterizzato dal numero di terminal presenti;
    \item \textbf{Aeroporto Regionale}, che presenta limitazioni operative, quali la lunghezza massima della pista e l'eventuale possibilità di operare in orario notturno;
    \item \textbf{Base Militare}, con codice NATO e livello minimo di sicurezza, oltre all'applicazione di specifici protocolli di sicurezza.
\end{itemize}
Un aeroporto può inoltre possedere diversi gate, che rappresentano i punti fisici di imbarco.

\subsection*{Aerei}
Ogni aereo è identificato da una matricola univoca e contiene:
\begin{itemize}
    \item data di immatricolazione;
    \item ore di volo accumulate.
\end{itemize}
Gli aerei appartengono a una gerarchia totale e sovrapposta, suddivisa in:
\begin{itemize}
    \item \textbf{Aereo di Linea}, dotato di rete Wi-Fi e configurazioni interne dedicate al trasporto passeggeri;
    \item \textbf{Aereo Cargo}, caratterizzato dal peso massimo di carico supportato;
    \item \textbf{Aereo Privato}, gestito per conto di un proprietario specifico, con eventuali servizi dedicati;
    \item \textbf{Aereo Militare}, utilizzato in operazioni governative, con un codice missione e un ente operativo.
\end{itemize}
Un aereo può svolgere contemporaneamente più ruoli (ad esempio: cargo privato, cargo militare).

\subsection*{Personale}
Il personale rappresenta tutte le persone che operano nel contesto aeroportuale. 
Per ciascun dipendente sono registrati:
\begin{itemize}
    \item identificativo univoco;
    \item nome e cognome;
    \item data di nascita;
    \item data di assunzione.
\end{itemize}
Il personale appartiene a una gerarchia parziale e disgiunta:
\begin{itemize}
    \item \textbf{Pilota}, con numero di licenza, ore di volo e data dell'ultima visita medica;
    \item \textbf{Controllore di Volo}, con livello e scadenza delle certificazioni abilitative;
    \item \textbf{Manutentore}, con qualifica tecnica e responsabilità di interventi e certificazioni.
\end{itemize}

\subsection*{Voli}
Il volo rappresenta il concetto centrale del sistema. Ogni volo è identificato da un id univoco e registra:
\begin{itemize}
    \item codice del volo;
    \item data e ora programmata di partenza;
    \item data e ora effettiva di partenza;
    \item stato operativo (programmato, in corso, atterrato, cancellato);
\end{itemize}
Il personale di bordo è gestito tramite la relazione di \textit{Equipaggio}, mentre i controllori di volo supervisionano le fasi critiche tramite la relazione \textit{Monitoraggio}.  
Ogni volo può prenotare specifiche fasce orarie per il decollo e l'atterraggio, 
tracciare i propri dati di telemetria tramite log dedicati e viene associato a una 
rotta pianificata. Un aereo esegue uno specifico volo.

\subsection*{Log di Transito}
I log di transito rappresentano i dati telemetrici generati durante un volo.  
Ogni rilevazione è associata a un singolo volo e a un istante temporale, e contiene 
informazioni quali altitudine e velocità.  
La gerarchia dei log è totale e disgiunta:
\begin{itemize}
    \item \textbf{Log Sensore GPS}, contenente latitudine e longitudine;
    \item \textbf{Log Dati di Volo}, con livello carburante, temperatura e pressione;
    \item \textbf{Log Errori}, con codice errore, messaggio e livello di severità.
\end{itemize}

\subsection*{Fascia Oraria}
Le fasce orarie rappresentano gli slot temporali gestiti dai singoli aereoporti destinati alle operazioni di decollo e atterraggio.  
Ogni fascia oraria contiene:
\begin{itemize}
    \item identificativo dello slot;
    \item orario di inizio e fine;
    \item tipologia dello slot.
\end{itemize}
I voli possono prenotare fasce di decollo e fasce di atterraggio distinte.

\subsection*{Rotta Pianificata}
Una rotta pianificata descrive un percorso standard e contiene:
\begin{itemize}
    \item identificativo della rotta;
    \item nome descrittivo;
    \item durata stimata;
    \item distanza;
    \item consumo previsto di carburante.
\end{itemize}
Ogni volo segue una specifica rotta.

\subsection*{Container}
I container rappresentano le unità di carico utilizzate per il trasporto merci.  
Per ciascun container sono registrati:
\begin{itemize}
    \item codice identificativo;
    \item peso della tara;
    \item tipologia.
\end{itemize}
Sono utilizzati sia per il trasporto tramite aerei cargo, sia per la registrazione delle 
merci imbarcate su un volo.

\subsection*{Classe Passeggeri e Servizi}
La classe passeggeri rappresenta una categoria di servizio a bordo degli aerei di linea, 
caratterizzata da:
\begin{itemize}
    \item identificativo della classe;
    \item spazio per le gambe.
\end{itemize}
A ciascuna classe sono associati eventuali \textit{Servizi Aggiuntivi}, ciascuno identificato da:
\begin{itemize}
    \item nome del servizio;
    \item classe a cui appartiene.
    \item descrizione.
\end{itemize}

\subsection*{Tipo di Aereo e Componenti}
Il tipo di aereo (modello) fornisce informazioni tecniche:
\begin{itemize}
    \item codice del modello;
    \item costruttore;
    \item autonomia;
    \item velocità di crociera.
\end{itemize}
Ogni modello può essere dotato di componenti tecnici specifici (entità debole), 
ciascuno caratterizzato da un numero di parte e una descrizione.

\subsection*{Compagnia Aerea}
Le compagnie aeree operano negli aeroporti e sono descritte da:
\begin{itemize}
    \item codice identificativo;
    \item nome;
    \item nazionalità;
    \item sito web.
\end{itemize}
Una compagnia può possedere diversi aerei e avere uno o più gate assegnati.

\subsection*{Gate}
I gate rappresentano i punti fisici di imbarco e sbarco presenti esclusivamente negli 
\textit{Hub Internazionali}. 
Si tratta di un'entità debole, poiché la loro esistenza dipende dall'hub cui appartengono.

Ogni gate registra:
\begin{itemize}
    \item numero del gate (univoco solo all'interno dello stesso hub);
    \item piano in cui si trova il gate;
    \item presenza o meno del tunnel di imbarco (jetbridge).
\end{itemize}

Alcuni gate possono essere assegnati a specifiche compagnie aeree per l’utilizzo operativo.



\subsection*{Protocollo di Sicurezza}
I protocolli di sicurezza si applicano alle basi militari e rappresentano le procedure operative 
obbligatorie da seguire. Ogni protocollo è caratterizzato da:
\begin{itemize}
    \item identificativo del protocollo;
    \item nome;
    \item descrizione.
\end{itemize}
Le basi militari possono applicare uno o più protocolli, ciascuno con una data di inizio validità.

\subsection*{Componenti}
I componenti rappresentano gli elementi tecnici standard installabili su uno specifico modello di aereo.  
Sono modellati come entità deboli, poiché identificati solo in relazione al tipo di aereo.

Ogni componente registra:
\begin{itemize}
    \item numero di parte (univoco all'interno del modello);
    \item descrizione tecnica;
    \item tipo di aereo (modello) a cui appartiene.
\end{itemize}
I componenti definiscono la dotazione tecnica minima associata a ogni modello di aereo.


\end{document}