\pdfminorversion=4
\documentclass[]{article}
\usepackage[utf8]{inputenc}
\usepackage{amssymb,latexsym,amsmath}
\usepackage[a4paper,top=3cm,bottom=2cm,left=3cm,right=3cm,marginparwidth=1.75cm]{geometry}
\usepackage{graphicx}
\usepackage[colorlinks=true, allcolors=blue]{hyperref}
\begin{document}

\section{Progettazione Logica}

Nel paragrafo che segue, parleremo del processo di ristrutturazione dello schema concettuale. L’obiettivo principale è rendere la rappresentazione dei dati più efficace, eliminando i valori nulli che non servono e preparando lo schema per la traduzione nel formato relazionale. Iniziamo con un’analisi delle ridondanze nel modello, per capire quali elementi tenere e quali è meglio togliere. Dopo, ci occuperemo di rimuovere le generalizzazioni. Alla fine, presenteremo lo schema E-R ristrutturato in base a questi criteri.

\subsection{Analisi delle Ridondanze}

\subsubsection*{Ridondanza 1: \textit{Numero\_Voli\_Giornalieri} in \texttt{Aeroporto}}
L’attributo \textit{NumeroVoliGiornalieri} nell’entità \texttt{Aeroporto} 
rappresenta il numero di voli (in partenza o in arrivo) associati ad uno specifico
aeroporto in una determinata data. Tale informazione può essere ottenuta
calcolando dinamicamente il numero di voli collegati tramite le relazioni
\texttt{Partenza(Aeroporto,Volo)} e \texttt{Arrivo(Aeroporto,Volo)}.

Tuttavia, il conteggio dei voli giornalieri comporta un costo elevato, poiché 
ogni aeroporto può avere migliaia di voli al giorno. Per questo motivo 
si valuta la convenienza di mantenere l’attributo ridondante
\textit{NumeroVoliGiornalieri} rispetto al ricalcolo continuo tramite query.

Consideriamo le seguenti operazioni tipiche:
\begin{itemize}
    \item \textbf{Operazione 1 (50 al giorno):} inserimento di un nuovo volo pianificato;
    \item \textbf{Operazione 2 (10\,000 al giorno):} consultazione del numero giornaliero di voli
    di un aeroporto.
\end{itemize}

I seguenti volumi sono considerati realistici per un sistema aeroportuale:
\begin{center}
\begin{tabular}{|c|c|c|}
\hline
Concetto & Tipo & Volume \\
\hline
Aeroporto & E & 120 \\
Volo & E & 80\,000 \\
Partenza/Arrivo & R & 160\,000 \\
\hline
\end{tabular}
\end{center}

\paragraph{Analisi con ridondanza}

L’inserimento di un nuovo volo comporta:
\begin{itemize}
    \item la creazione dell’entità \texttt{Volo};
    \item l’inserimento delle due relazioni \texttt{Partenza} e \texttt{Arrivo};
    \item la lettura e l’aggiornamento dell’attributo ridondante \texttt{NumeroVoliGiornalieri}
          nell’entità \texttt{Aeroporto}.
\end{itemize}

\begin{table}[h!]
\centering
\begin{tabular}{|c|c|c|c|}
\hline
Concetto & Costrutto & Accessi & Tipo \\
\hline
Volo & E & 1 & S $\times$ 50 \\
Partenza & R & 1 & S $\times$ 50 \\
Arrivo & R & 1 & S $\times$ 50 \\
Aeroporto & E & 1 & L $\times$ 50 \\
Aeroporto & E & 1 & S $\times$ 50 \\
\hline
\end{tabular}
\caption{Operazione 1 con ridondanza: inserimento nuovo volo.}
\end{table}

Per la consultazione del numero di voli giornalieri:
\begin{table}[h!]
\centering
\begin{tabular}{|c|c|c|c|}
\hline
Concetto & Costrutto & Accessi & Tipo \\
\hline
Aeroporto & E & 1 & L $\times$ 10\,000 \\
\hline
\end{tabular}
\caption{Operazione 2 con ridondanza: consultazione del numero giornaliero di voli.}
\end{table}

\paragraph{Analisi senza ridondanza}

In assenza dell’attributo ridondante, la consultazione giornaliera richiede 
di contare i voli connessi tramite \texttt{Partenza} e \texttt{Arrivo}.  
In media un aeroporto gestisce circa 800 voli al giorno.

L’inserimento di un nuovo volo richiede soltanto la registrazione
dell’entità \texttt{Volo} e delle relazioni \texttt{Partenza} e \texttt{Arrivo}:

\begin{table}[h!]
\centering
\begin{tabular}{|c|c|c|c|}
\hline
Concetto & Costrutto & Accessi & Tipo \\
\hline
Volo & E & 1 & S $\times$ 50 \\
Partenza & R & 1 & S $\times$ 50 \\
Arrivo & R & 1 & S $\times$ 50 \\
\hline
\end{tabular}
\caption{Operazione 1 senza ridondanza.}
\end{table}

Per la consultazione:

\begin{table}[h!]
\centering
\begin{tabular}{|c|c|c|c|}
\hline
Concetto & Costrutto & Accessi & Tipo \\
\hline
Aeroporto & E & 1 & L $\times$ 10\,000 \\
Partenza & R & 800 & L $\times$ 10\,000 \\
Arrivo & R & 800 & L $\times$ 10\,000 \\
\hline
\end{tabular}
\caption{Operazione 2 senza ridondanza: conteggio dei voli giornalieri.}
\end{table}

\paragraph{Conclusione.}  
La consultazione quotidiana, pari a 10\,000 accessi, comporterebbe senza ridondanza
circa $16$ milioni di letture, rendendo l'operazione estremamente costosa.  
L’analisi suggerisce pertanto di \textbf{mantenere l’attributo ridondante NumeroVoliGiornalieri},
che ottimizza sensibilmente le interrogazioni frequenti a scapito di un modesto 
incremento nel costo di scrittura.

\subsubsection*{Ridondanza 2: \textit{Peso\_Totale\_Carico} in \texttt{Volo}}

Il peso totale delle merci caricate su un volo è un valore derivabile dalla relazione
\texttt{Imbarco\_Merce}. Per ogni coppia \((c,v)\) (container–volo), tale relazione
registra il valore \textit{Peso\_Registrato}. Pertanto:

\[
\text{Peso\_Totale\_Carico}(v) = 
\sum_{\,c \in \text{Container}(v)} \text{Peso\_Registrato}(c,v)
\]

Si valuta l'introduzione dell'attributo ridondante \textit{Peso\_Totale\_Carico}
nell'entità \texttt{Volo}, aggiornato a ogni operazione di imbarco o sbarco.

Consideriamo le operazioni tipiche:
\begin{itemize}
    \item \textbf{Operazione 1 (circa 50 al giorno):} inserimento o rimozione di un container da un volo;
    \item \textbf{Operazione 2 (circa 2\,000 al giorno):} consultazione del peso totale del carico di un volo.
\end{itemize}

Assumiamo costi di accesso: L=1, S=2.

I volumi rilevanti del sistema sono:
\begin{center}
\begin{tabular}{|c|c|c|}
\hline
Concetto & Tipo & Volume \\
\hline
Volo & E & 8\,000 \\
Container & E & 40\,000 \\
Imbarco\_Merce & R & 60\,000 \\
\hline
\end{tabular}
\end{center}

\paragraph{Senza ridondanza.}
Ogni consultazione del peso richiede di:
\begin{itemize}
    \item leggere il volo (1L);
    \item leggere tutti i container imbarcati: in media 20 per volo (20L);
    \item effettuare una somma su 20 valori.
\end{itemize}

\begin{table}[h!]
\centering
\begin{tabular}{|c|c|c|c|}
\hline
Concetto & Costrutto & Accessi & Tipo \\
\hline
Imbarco\_Merce & R & 1 & S $\times$ 50 \\
\hline
\end{tabular}
\caption{Operazione 1 senza ridondanza.}
\end{table}

\begin{table}[h!]
\centering
\begin{tabular}{|c|c|c|c|}
\hline
Concetto & Costrutto & Accessi & Tipo \\
\hline
Volo & E & 1 & L $\times$ 2\,000 \\
Imbarco\_Merce & R & 20 & L $\times$ 2\,000 \\
\hline
\end{tabular}
\caption{Operazione 2 senza ridondanza: scansione dei container imbarcati.}
\end{table}

Costo giornaliero senza ridondanza:
\[
2\,000 \times (1 + 20) = 42\,000.
\]

\paragraph{Con ridondanza.}
L'operazione di aggiornamento del peso totale richiede:
\begin{itemize}
    \item aggiornare \texttt{Imbarco\_Merce} (1S);
    \item leggere \texttt{Volo} (1L) e aggiornarne \textit{Peso\_Totale\_Carico} (1S).
\end{itemize}

\begin{table}[h!]
\centering
\begin{tabular}{|c|c|c|c|}
\hline
Concetto & Costrutto & Accessi & Tipo \\
\hline
Imbarco\_Merce & R & 1 & S $\times$ 50 \\
Volo & E & 1 & L $\times$ 50 \\
Volo & E & 1 & S $\times$ 50 \\
\hline
\end{tabular}
\caption{Operazione 1 con ridondanza: aggiornamento del peso totale.}
\end{table}

Per la consultazione del peso totale:
\begin{table}[h!]
\centering
\begin{tabular}{|c|c|c|c|}
\hline
Concetto & Costrutto & Accessi & Tipo \\
\hline
Volo & E & 1 & L $\times$ 2\,000 \\
\hline
\end{tabular}
\caption{Operazione 2 con ridondanza: lettura diretta del valore.}
\end{table}

\[
\text{Costo giornaliero} =
50 \times (2 + 1 + 2) + 2\,000 \times 1
= 250 + 2\,000 = 2\,250.
\]

\paragraph{Conclusione.}
Senza ridondanza, la consultazione comporta circa 42\,000 accessi al giorno, dovuti
alla scansione dei container imbarcati. Con la ridondanza il costo scende a circa
2\,150 accessi. L’analisi indica chiaramente che conviene mantenere l’attributo 
ridondante \textit{Peso\_Totale\_Carico} nell’entità \texttt{Volo}.


\subsection{Eliminazione delle Generalizzazioni}

In questa sezione si eliminano le generalizzazioni identificate 
nella Progettazione Concettuale. L'obiettivo è quello di arrivare a uno schema che non contenga costrutti che non si possano tradurre in un modello relazionale. Togliere le generalizzazioni aiuta a ridurre gli attributi che non si possono applicare e assicura una gestione corretta delle chiavi e dei vincoli di integrità.

Le generalizzazioni presenti nel nostro schema riguardano:
\begin{itemize}
    \item \textbf{Aeroporto}, suddiviso in Hub Internazionale, Aeroporto Regionale e Base Militare;
    \item \textbf{Aereo}, suddiviso in Aereo di Linea, Aereo Cargo, Aereo Privato e Aereo Militare;
    \item \textbf{Personale}, suddiviso in Pilota, Controllore di Volo e Manutentore;
    \item \textbf{Log di Transito}, suddiviso in Log Dati di Volo, Log Sensori GPS e Log Errori.
\end{itemize}

Le modalità con cui tali generalizzazioni vengono eliminate sono descritte qui di seguito.

\paragraph{Aeroporto}
La generalizzazione su \textbf{Aeroporto} è \emph{totale e disgiunta}, 
dato che ogni aeroporto può rientrare solo in una specifica categoria. La generalizzazione viene eliminata creando tre relazioni distinte:
\texttt{IS\_HUB}, \texttt{IS\_REG}, \texttt{IS\_MIL}.  
In queste relazioni ogni occorrenza è identificata tramite la stessa chiave primaria 
dell’entita \texttt{Aeroporto}.  
Questa trasformazione evita la presenza di attributi nulli: ad esempio, 
gli aeroporti regionali non devono contenere attributi relativi ai protocolli di sicurezza militari, e le basi militari non devono registrare il numero di terminal tipici degli hub internazionali.

\paragraph{Aereo}
La generalizzazione su \textbf{Aereo} è \emph{totale e sovrapposta}, 
poiche un aeromobile può svolgere più ruoli operativi (ad esempio, 
aereo cargo impiegato temporaneamente come velivolo militare).  
Poprio a causa della natura sovrapposta, non è possibile rimuovere l’entità padre 
senza duplicare la chiave nelle entità figlie, il che porterebbe a introdurre incoerenze.
Perciò la generalizzazione viene gestita tramite relazioni dedicate:
\texttt{IS\_LINEA}, \texttt{IS\_CARGO}, \texttt{IS\_PRIV}, \texttt{IS\_MIL},
ciascuna referenziante la chiave primaria dell’entità \texttt{Aereo}.  
Questa soluzione ci consente di associare un aereo a più categorie allo stesso tempo,
mantenendo un modello relazionale compatto e rispettando i requisiti funzionali.

\paragraph{Personale}
La generalizzazione su \textbf{Personale} è \emph{parziale e disgiunta}.  
Non tutte le persone sono piloti, controllori o manutentori, e un membro del personale
non può appartenere a più di una di queste categorie specializzate.  
La rimozione avviene introducendo tre relazioni:
\texttt{IS\_PILOTA}, \texttt{IS\_CTRL}, \texttt{IS\_MAN}.  
Poichè si tratta di una generalizzazione parziale, l'entità padre \texttt{Personale} non può essere eliminata,
visto che alcune persone non appartengono a nessun ruolo specializzato.

\paragraph{Log di Transito}
La generalizzazione su \textbf{LogTransito} è \emph{totale e disgiunta}.  
Ogni rilevazione di log deve rientrare necessariamente in una sola delle categorie stabilite.
La generalizzazione viene rimossa introducendo tre relazioni figlie:
\texttt{IS\_GPS}, \texttt{IS\_DATIVOLO}, \texttt{IS\_ERRORE}.  
L’entità padre \texttt{LogTransito} resta, poiché include gli attributi generali
(altitudine, velocità, timestamp) e svolge il ruolo di contenitore dei dati comuni.

\medskip
\noindent
Il diagramma E-R ristrutturato tiene conto di queste modifiche.

\section*{Schema Relazionale}

Lo schema ristrutturato contiene esclusivamente costrutti direttamente 
mappabili nel modello relazionale.  
Di seguito viene riportato lo schema logico completo, dove l’asterisco (*)
denota gli attributi che possono assumere valori nulli.

\begin{itemize}

\item \textbf{Aeroporto}(\underline{codice}, nome, citta, nazione, coordinate, numero\_piste, numero\_voli\_giornalieri)

\item \textbf{HubInternazionale}(\underline{codice}, numero\_terminal)  
\hspace{3mm} codice $\rightarrow$ Aeroporto(codice)

\item \textbf{Regionale}(\underline{codice}, lunghezza\_max\_pista, notturno)  
\hspace{3mm} codice $\rightarrow$ Aeroporto(codice)

\item \textbf{BaseMilitare}(\underline{codice}, codice_nato, livello_sicurezza)  
\hspace{3mm} codice $\rightarrow$ Aeroporto(codice)

\item \textbf{Gate}(\underline{codice}, \underline{numero_gate}, piano, tunnel)  
\hspace{3mm} codice $\rightarrow$ Aeroporto(codice)

\item \textbf{Aereo}(\underline{matricola}, data_immatricolazione, ore_volo_accumulate)

\item \textbf{AereoDiLinea}(\underline{matricola}, wifi*, intrattenimento*)  
\hspace{3mm} matricola $\rightarrow$ Aereo(matricola)

\item \textbf{AereoCargo}(\underline{matricola}, volume_carico_max)  
\hspace{3mm} matricola $\rightarrow$ Aereo(matricola)

\item \textbf{AereoPrivato}(\underline{matricola}, proprietario, servizi_inclusi*)  
\hspace{3mm} matricola $\rightarrow$ Aereo(matricola)

\item \textbf{AereoMilitare}(\underline{matricola}, codice_missione, ente_operativo)  
\hspace{3mm} matricola $\rightarrow$ Aereo(matricola)

\item \textbf{TipoAereo}(\underline{id_tipo}, costruttore, autonomia, velocita_crociera)

\item \textbf{Componente}(\underline{id_tipo}, \underline{numero_parte}, descrizione)  
\hspace{3mm} id_tipo $\rightarrow$ TipoAereo(id_tipo)

\item \textbf{Volo}(\underline{id_volo}, codice_volo, data_ora_prog, data_ora_eff*, stato, peso_totale_carico, matricola_aereo) \\
\hspace{3mm} matricola_aereo $\rightarrow$ Aereo(matricola)

\item \textbf{Partenza}(id_volo, aeroporto_origine)  
\hspace{3mm} id_volo $\rightarrow$ Volo(id_volo)  
\hspace{3mm} aeroporto_origine $\rightarrow$ Aeroporto(codice)

\item \textbf{Arrivo}(id_volo, aeroporto_destinazione)  
\hspace{3mm} id_volo $\rightarrow$ Volo(id_volo)  
\hspace{3mm} aeroporto_destinazione $\rightarrow$ Aeroporto(codice)

\item \textbf{FasciaOraria}(\underline{id_slot}, ora_inizio, ora_fine, tipologia, stato_slot)

\item \textbf{PrenotazioneDecollo}(id_volo, id_slot)  
\hspace{3mm} id_volo $\rightarrow$ Volo(id_volo)  
\hspace{3mm} id_slot $\rightarrow$ FasciaOraria(id_slot)

\item \textbf{PrenotazioneAtterraggio}(id_volo, id_slot)  
\hspace{3mm} id_volo $\rightarrow$ Volo(id_volo)  
\hspace{3mm} id_slot $\rightarrow$ FasciaOraria(id_slot)

\item \textbf{RottaPianificata}(\underline{id_rotta}, nome, distanza, durata_stimata, consumo_previsto)

\item \textbf{AssegnazioneRotta}(id_volo, id_rotta)  
\hspace{3mm} id_volo $\rightarrow$ Volo(id_volo)  
\hspace{3mm} id_rotta $\rightarrow$ RottaPianificata(id_rotta)

\item \textbf{Personale}(\underline{id_personale}, nome, cognome, data_nascita, data_assunzione)

\item \textbf{Pilota}(\underline{id_personale}, numero_licenza, ore_volo, ultima_visita)  
\hspace{3mm} id_personale $\rightarrow$ Personale(id_personale)

\item \textbf{ControlloreVolo}(\underline{id_personale}, livello_certificazione, scadenza_abilitazione)  
\hspace{3mm} id_personale $\rightarrow$ Personale(id_personale)

\item \textbf{Manutentore}(\underline{id_personale}, qualifica)  
\hspace{3mm} id_personale $\rightarrow$ Personale(id_personale)

\item \textbf{Equipaggio}(id_personale, id_volo, ruolo, ore_servizio) \\
\hspace{3mm} id_personale $\rightarrow$ Personale(id_personale) \\
\hspace{3mm} id_volo $\rightarrow$ Volo(id_volo)

\item \textbf{Monitoraggio}(id_personale, id_volo, fase_volo*) \\
\hspace{3mm} id_personale $\rightarrow$ ControlloreVolo(id_personale) \\
\hspace{3mm} id_volo $\rightarrow$ Volo(id_volo)

\item \textbf{Presidio}(id_personale, codice, data_inizio, data_fine*, settore) \\
\hspace{3mm} id_personale $\rightarrow$ ControlloreVolo(id_personale) \\
\hspace{3mm} codice $\rightarrow$ Aeroporto(codice)

\item \textbf{Container}(\underline{codice_container}, peso_tara, tipologia)

\item \textbf{ImbarcoMerce}(id_volo, codice_container, peso_registrato, data_imbarco) \\
\hspace{3mm} id_volo $\rightarrow$ Volo(id_volo) \\
\hspace{3mm} codice_container $\rightarrow$ Container(codice_container)

\item \textbf{Trasporto}(matricola, codice_container, data_carico, posizione_stiva) \\
\hspace{3mm} matricola $\rightarrow$ AereoCargo(matricola) \\
\hspace{3mm} codice_container $\rightarrow$ Container(codice_container)

\item \textbf{ClassePasseggeri}(\underline{id_classe}, spazio_gambe)

\item \textbf{Configurazione}(matricola, id_classe, numero_posti) \\
\hspace{3mm} matricola $\rightarrow$ AereoDiLinea(matricola) \\
\hspace{3mm} id_classe $\rightarrow$ ClassePasseggeri(id_classe)

\item \textbf{ServizioAggiuntivo}(\underline{id_classe}, \underline{nome_servizio}, descrizione)  
\hspace{3mm} id_classe $\rightarrow$ ClassePasseggeri(id_classe)

\item \textbf{LogTransito}(\underline{id_log}, id_volo, timestamp, altitudine*, velocita*) \\
\hspace{3mm} id_volo $\rightarrow$ Volo(id_volo)

\item \textbf{LogGPS}(\underline{id_log}, latitudine, longitudine) \\
\hspace{3mm} id_log $\rightarrow$ LogTransito(id_log)

\item \textbf{LogDatiVolo}(\underline{id_log}, livello_carburante, pressione, temperatura) \\
\hspace{3mm} id_log $\rightarrow$ LogTransito(id_log)

\item \textbf{LogErrori}(\underline{id_log}, codice_errore, messaggio, severita) \\
\hspace{3mm} id_log $\rightarrow$ LogTransito(id_log)

\item \textbf{GestioneAlert}(id_personale, id_log, stato, data_presa_in_carico) \\
\hspace{3mm} id_personale $\rightarrow$ Manutentore(id_personale) \\
\hspace{3mm} id_log $\rightarrow$ LogErrori(id_log)

\end{itemize}


\end{document}