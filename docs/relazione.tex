\documentclass[10pt]{article}
\usepackage[utf8]{inputenc}
\usepackage{amssymb,latexsym,amsmath}
\usepackage[a4paper,top=3cm,bottom=2cm,left=3cm,right=3cm,marginparwidth=1.75cm]{geometry}
\usepackage[scaled=0.92]{helvet}
\renewcommand\familydefault{\sfdefault}
\usepackage[T1]{fontenc}
\usepackage{graphicx}
\usepackage{float}
\usepackage[colorlinks=true, allcolors=blue]{hyperref}
\usepackage{listings}
\usepackage{xcolor}
\lstdefinestyle{sqlstyle}{
    language=SQL,
    basicstyle=\ttfamily\small,
    keywordstyle=\color{blue}\bfseries,
    commentstyle=\color{gray},
    stringstyle=\color{teal},
    showstringspaces=false,
    breaklines=true,
    frame=lines
}

\begin{document}
\section{Abstract}
Il progetto \textbf{STAR} (Sistema di Tracciamento dello Spazio Aereo e Rotte) è un Sistema Informativo pensato per gestire e monitorare, in modo integrato, il traffico aereo sia civile che militare. Questo sistema segue tutte le fasi operative di un volo, dalla pianificazione della rotta e prenotazione degli slot negli aeroporti, fino alla storicizzazione delle operazioni effettuate e alla gestione dei dati telemetrici


La banca dati si concentra su tre aree principali:

\begin{itemize}
    \item \textbf{Infrastruttura e Logistica}: Qui si gestiscono gli aeroporti, sia hub che regionali e militari, i gate e tutta la complessa logistica di carico merce (container) e passeggeri.
    
    \item \textbf{Flotta e Risorse Umane}: Questo aspetto si occupa in dettaglio degli aeromobili, con tutte le loro configurazioni, e del personale specializzato come piloti, controllori e manutentori, tenendo traccia delle loro abilitazioni, turni e certificazioni.

    
    \item \textbf{Sicurezza e Monitoraggio}: Archiviazione dei log operativi (rilevazioni GPS campionate, parametri tecnici e messaggi di errore) finalizzata alla manutenzione predittiva e alla gestione dei protocolli di sicurezza per le aree militari.
\end{itemize}

L'obiettivo è quello di supportare le decisioni per ottimizzare le risorse aeroportuali e garantire una completa tracciabilità degli eventi di volo e di manutenzione.


\section{Analisi dei Requisiti}

Questa sezione descrive i concetti informativi rilevanti del sistema e gli attributi principali
associati a ciascuna entità.

\subsection*{Aeroporti}
Un aeroporto è identificato da un codice IATA e presenta: nome, città, nazione, coordinate GPS,
numero di piste e numero di voli giornalieri. Si registrano inoltre le informazioni relative ai
controllori di volo assegnati (settore, data e ora di inizio e fine turno). Gli aeroporti sono classificati in
Hub Internazionali, con numero di terminal; Aeroporti Regionali, con lunghezza massima della
pista e possibilità di operare in orario notturno; Basi Militari, con codice NATO e livello minimo
di sicurezza. Gli Hub possono includere più gate, mentre gli aeroporti Regionali possono avere limitazioni sulla tipologia di aereo

\subsection*{Aerei}
Ogni aereo è identificato da una matricola e registra data di immatricolazione e ore totali di volo.
Gli aerei sono classificati in Aerei di Linea (con servizi come Wi-Fi e intrattenimento), Aerei Cargo
(con un peso massimo di carico), Aerei Privati (con proprietario e servizi dedicati) e Aerei
Militari (con codice missione ed ente operativo).

\subsection*{Personale}
Per ogni membro del personale sono registrati identificativo, nome, cognome, data di nascita e
data di assunzione. Le specializzazioni includono: il Pilota, con numero di licenza, ore di volo,
ultima visita e abilitazioni di volo per i vari modelli di aereo (le quali includono la data di conseguimento e la data di scadenza);
Controllore di Volo, che possiede il livello di certificazione e la scadenza dell’abilitazione; 
e il Manutentore, con qualifica e possibilità di effettuare manutenzioni e quindi rilasciare certificati di avvenuta manutenzione. Questi certificati hanno una data di scadenza.
I Manutentori inoltre, gestiscono gli alert generati dai log di Errore, i quali presentano uno stato (creato, risolto, ..) e una data di presa in carico.
\subsection*{Voli}
Ogni volo è descritto mediante identificativo, codice del volo, data e ora di partenza programmata,
data e ora di partenza effettiva, stato operativo e peso totale del carico. Si registrano i membri
dell’equipaggio (ruolo, ore di servizio), le fasi di volo monitorate dai controllori, le fasce orarie prenotate
per decollo e atterraggio e la rotta pianificata. Ogni volo è eseguito da un singolo aereo.

\subsection*{Log di Transito}
Per ogni rilevazione telemetrica sono registrati identificativo del volo che ha generato il Log, timestamp, altitudine e velocità.
Sono previsti tre tipi di log: rilevazioni GPS (latitudine e longitudine), dati di volo (livello
carburante, temperatura, pressione) ed errori (codice, messaggio e severità), utilizzati nella gestione
operativa della manutenzione.

\subsection*{Fascia Oraria}
Ogni slot operativo comprende l'orario di inizio e fine e tipologia. I voli possono
prenotare separatamente gli slot di decollo e di atterraggio. Ogni fascia oraria si intende di uno specifico aeroporto, classificata tramite identificativo dello slot.

\subsection*{Rotta Pianificata}
Una rotta pianificata è descritta da identificativo, nome, durata stimata, distanza e consumo
previsto di carburante, oltre all'aeroporto di origine e di destinazione.

\subsection*{Container}
Ogni container registra codice identificativo, peso tara e tipologia. Si conservano inoltre le
informazioni relative alla data di carico e ai dati relativi al peso registrato e
al contenuto quando imbarcato su un volo.

\subsection*{Classe Passeggeri e Servizi}
Le classi passeggeri presentano un identificativo, il nome commerciale (es. Economy), la priorità di imbarco, e una descrizione. In ogni configurazione di aereo di linea, hanno un numero di posti e lo spazio per le gambe offerto. Ogni classe Passeggeri può offrire un servizio aggiuntivo, compreso di nome e descrizione.

\subsection*{Tipo di Aereo e Componenti}
Ogni modello di aereo presenta codice, costruttore, autonomia e velocità di crociera. I componenti
tecnici associati a ciascun modello sono descritti tramite numero di parte e descrizione.

\subsection*{Compagnia Aerea}
Ogni compagnia aerea è caratterizzata da codice identificativo, nome, nazionalità e sito web. Può
possedere più aerei e avere uno o più gate assegnati.

\subsection*{Gate}
I gate degli hub internazionali sono descritti tramite numero, piano e presenza del tunnel di
imbarco. Possono essere assegnati a specifiche compagnie aeree.

\subsection*{Protocollo di Sicurezza}
Ogni protocollo applicabile alle basi militari registra identificativo, nome e descrizione, oltre alla
data di inizio validità.


\section{Progettazione concettuale}

\begin{figure}[H] 
    \centering
    \includegraphics[width=\textwidth]{schema_er.drawio.png}
    \caption{Diagramma Entità-Relazione del progetto STAR}
    \label{fig:schema_er}
\end{figure}


La Figura 1 mostra il diagramma Entità-Relazione (E-R) creato sulla base dell'analisi dei requisiti. Il sistema rappresenta una rete aeroportuale, suddivisa in tre categorie: Hub Internazionali, Aeroporti Regionali e Basi Militari. Questa classificazione crea una gerarchia totale e disgiunta, poichè ogni aeroporto deve appartenere esclusivamente a una di queste categorie. Gli Hub Internazionali includono diversi Gate, che sono modellati come entità deboli e identificati dalla coppia (codice\_IATA, numero). Le specifiche tecniche di ciascun aereo sono gestite dall'entità Tipo Aereo, che rappresenta il catalogo degli aerei del sistema e sono organizzati tramite una generalizzazione totale e sovrapposta, con le categorie Aerei di Linea, Aerei Cargo, Aerei Privati e Aerei Militari. Un velivolo può avere più ruoli contemporaneamente, come nel caso di configurazioni miste passeggeri-cargo o di freighter usati in ambito militare. Gli Aerei di Linea hanno anche una Configurazione interna, definita rispetto all'entità Classe Passeggeri, che rappresenta le varie categorie di servizio disponibili a bordo (ad esempio, Economy o Business). A ciascuna classe possono essere associati uno o più Servizi Aggiuntivi, modellati come entità deboli indicate dalla coppia (id\_classe, nome\_servizio). L'entità Volo rappresenta la gestione di tutti i singoli voli, ed è l'evento centrale del sistema. Le relazioni di Partenza e Arrivo collegano ogni volo rispettivamente all'aeroporto di partenza e a quello di destinazione. Anche se la rotta pianificata indica l'itinerario previsto, mantenere relazioni dirette sul volo è essenziale per registrare l'evento reale, assicurando la coerenza del dato anche in caso di variazioni operative (come dirottamenti o atterraggi di emergenza) o per voli specifici (militari o privati) che non seguono una rotta standard. Ogni volo è effettuato da un singolo aereo tramite la relazione Esegue ed è sorvegliato da uno o più Controllori di Volo attraverso la relazione Monitoraggio. L'entità Personale descrive le risorse umane coinvolte nelle operazioni aeroportuali ed è organizzata in una generalizzazione parziale e disgiunta nelle sottoclassi Pilota, Controllore di Volo e Manutentore. Durante un volo, si registrano i Log di Transito, cioè entità deboli identificate dalla coppia (id\_volo, timestamp\_rilevazione). Questa gerarchia è totale e disgiunta e comprende tre tipi di log: Log Sensore GPS, Log Dati di Volo e Log Errori. I log d'errore possono generare notifiche per i manutentori tramite la relazione Gestione Alert. Per quanto riguarda la gestione delle merci, l'entità Container rappresenta le unità di carico trasportate nel sistema. La relazione Imbarco Merce associa ogni container ai voli sui quali viene caricato, registrando peso e contenuto dell'operazione. L'entità Rotta Pianificata descrive i collegamenti standard tra aeroporti e viene associata ai voli tramite la relazione Assegnazione Rotta. La Tabella 1 riassume tutte le entità e relazioni illustrate nello schema E-R, indicando attributi e identificatori.

\subsection*{Vincoli non rappresentabili nello schema E-R}

Alcune regole del dominio non possono essere rappresentate direttamente nel diagramma
E--R, ma costituiscono comunque vincoli fondamentali del sistema. Tra i principali:

\begin{itemize}

   \item Un aereo classificato come Militare non può essere posseduto da una compagnia aerea commerciale. Lo schema consente la relazione Possesso con qualsiasi aereo, ma il dominio richiede il vincolo:
\[
a \in AereoMilitare \Rightarrow \neg \exists c \, Possesso(c, a)
\]

\item Un container può essere imbarcato su un volo solo se il peso registrato non supera la capacità di carico dell’aereo che esegue quel volo. Questo vincolo incrocia attributi e relazioni diverse (ImbarcoMerce, Esegue, AereoCargo) ed è quindi non rappresentabile graficamente:
\[
ImbarcoMerce(c, v) \Rightarrow peso\_registrato(c, v) \leq peso\_carico\_max(Esegue(v))
\]

\item Un controllore di volo può monitorare un volo solo se è assegnato a un aeroporto coerente con quel volo. In particolare, se un controllore monitora un volo, deve presidiare almeno uno degli aeroporti di partenza o arrivo:
\[
Monitoraggio(p, v) \Rightarrow \exists a \, Presidio(p, a) \wedge (a = Partenza(v) \vee a = Arrivo(v))
\]

\item La gestione degli errori dipende dallo stato del ticket. In particolare, uno stato diverso da "non assegnato" richiede che esista almeno un manutentore associato tramite Gestione Alert. Il legame tra stato del log e presenza di un manutentore non è esprimibile nel solo schema E-R.

\item Un membro dell’equipaggio può operare come Pilota solo se è abilitato al modello dell’aereo che esegue il volo. Questo vincolo incrocia Equipaggio, Pilota, AbilitazioneVolo, Esegue e TipoAereo:
\[
Equipaggio(p, v) \wedge Ruolo(p, v) = "Pilota" \Rightarrow \exists t \, AbilitazioneVolo(p, t) \wedge t = TipoAereo(Esegue(v))
\]

\item \textbf{Coerenza tra Rotta e Volo.} Se a un volo è assegnata una rotta pianificata, gli aeroporti di partenza e arrivo registrati nel volo devono coincidere con l'origine e la destinazione previste dalla rotta (salvo eccezioni operative gestite a livello applicativo).
\[
AssegnazioneRotta(v, r) \Rightarrow (Partenza(v) = Origine(r)) \wedge (Arrivo(v) = Destinazione(r))
\]

\item \textbf{Coerenza Slot-Aeroporto.} Un volo può prenotare uno slot (di decollo o atterraggio) solo se tale fascia oraria appartiene all'aeroporto previsto per quell'operazione. Ad esempio, per il decollo:
\[
PrenotazioneDecollo(v, s) \wedge FasciaOraria(s, a) \Rightarrow Partenza(v) = a
\]

\end{itemize}

\begin{table}[H]
\centering
\begin{tabular}{|p{4cm}|p{4cm}|p{4cm}|p{4cm}|}
\hline
\textbf{Entità} & \textbf{Descrizione} & \textbf{Attributi} & \textbf{Identificatore} \\
\hline

Aeroporto & Struttura aeroportuale & codice\_IATA, nome, città, nazione, coordinate, numero\_piste, numero\_voli\_giornalieri& codice\_IATA\\
\hline

HubInternazionale & Tipo di aeroporto & numero\_terminal & \\
\hline

Regionale & Tipo di aeroporto & lunghezza\_pista\_max, notturno& \\
\hline

BaseMilitare & Aeroporto militare & codice\_nato, livello\_sicurezza& \\
\hline

Gate& Punto di imbarco & numero, piano, tunnel\_imbarco& numero, Relazione "Possiede"\\
\hline

ProtocolloSicurezza & Procedure militari & id, nome, descrizione& id\\
\hline

Aereo & Velivolo & matricola, data\_immatricolazione, ore\_volo\_accumulate& matricola\\
\hline

AereoDiLinea & Sottotipo aereo & wifi, intrattenimento& \\
\hline

AereoCargo & Sottotipo aereo & peso\_carico\_max& \\
\hline

AereoPrivato & Sottotipo aereo & nome\_proprietario, servizio& \\
\hline

AereoMilitare & Sottotipo aereo & codice\_missione, ente\_operativo& \\
\hline

TipoAereo & Catalogo modelli & codice\_modello, costruttore, autonomia, velocita\_crociera& codice\_modello\\
\hline

Componente& Componenti tecnici standard & numero, descrizione & numero, Relazione "Dotazione"\\
\hline

Personale & Risorse umane & id, nome, cognome, data\_nascita, data\_assunzione& id\\
\hline

Pilota & Sottoclasse personale & numero\_licenza, ore\_volo\_totali, ultima\_visita\_medica& \\
\hline

ControlloreVolo & Sottoclasse personale & livello\_certificazione, scadenza\_abilitazione& \\
\hline

Manutentore & Sottoclasse personale & qualifica& \\
\hline

Volo & Evento operativo & id\_volo, codice\_volo, data\_programmata, data\_partenza\_effettiva, stato, peso\_totale\_carico& id\_volo\\
\hline

LogTransito& Rilevazioni telemetriche & timestamp\_rilevazione, altitudine, velocità& timestamp\_rilevazione, Relazione "Traccia"\\
\hline

LogGPS & Sottotipo log & latitudine, longitudine& \\
\hline

LogDatiVolo & Sottotipo log & livello\_carburante, temperatura\_esterna, pressione\_cabina& \\
\hline

LogErrori & Sottotipo log & codice\_errore, messaggio, livello\_severità& \\
\hline

FasciaOraria & Slot aeroportuali & id\_slot, orario\_inizio, orario\_fine, tipo& id\_slot\\
\hline

RottaPianificata & Percorso standard & id\_rotta, nome\_rotta, consumo\_carburante, distanza, durata\_stimata& id\_rotta\\
\hline

Container & Unità di carico & codice\_container, peso\_tara, tipologia& codice\_container\\
\hline

ClassePasseggeri & Catalogo Classi di viaggio & id\_classe, nome\_commerciale, priorità\_imbarco, descrizione& id\_classe\\
\hline

ServizioAggiuntivo& Servizi associati alle classi & nome\_servizio, descrizione& nome\_servizio, Relazione "Offre"\\
\hline

CompagniaAerea & Operatore aeroportuale & codice, nome, nazione, sito\_web& codice\\
\hline

\end{tabular}
\caption{Entità individuate dall’analisi dei requisiti, con attributi e identificatori.}
\end{table}

\begin{table}[H]
\centering
\begin{tabular}{|p{4cm}|p{4cm}|p{4cm}|p{4cm}|}
\hline
\textbf{Relazione} & \textbf{Descrizione} & \textbf{Componente} & \textbf{Attributi} \\
\hline

Possiede & Un Hub Internazionale possiede uno o più gate & Hub internazionale, Gate & -- \\
\hline

Applica & Una base militare applica protocolli di sicurezza & BaseMilitare, ProtocolloSicurezza & data\_inizio\_validità \\
\hline

DivietoOperativo & Un aeroporto regionale vieta certi tipi di aereo & Regionale, TipoAereo & -- \\
\hline

Esecuzione & Un aereo esegue un volo & Aereo, Volo & -- \\
\hline

Partenza & Un volo parte da un aeroporto & Volo, Aeroporto & -- \\
\hline

Arrivo & Un volo arriva in un aeroporto & Volo, Aeroporto & -- \\
\hline

AssegnazioneRotta & Una rotta viene associata a un volo & RottaPianificata, Volo & -- \\
\hline

PrenotazioneDecollo & Un volo prenota uno slot di decollo & Volo, FasciaOraria & -- \\
\hline

PrenotazioneAtterraggio & Un volo prenota uno slot di atterraggio & Volo, FasciaOraria & -- \\
\hline

Equipaggio & Piloti e assistenti operano su un volo & Personale, Volo & ruolo, ore\_servizio \\
\hline

Monitoraggio & Un controllore supervisiona un volo & ControlloreVolo, Volo & fase\_volo \\
\hline

Presidio & Un controllore è assegnato a un aeroporto & ControlloreVolo, Aeroporto & settore, data\_fine, data\_inizio \\
\hline

Traccia & Un volo genera log di transito & Volo, LogTransito & -- \\
\hline

GestioneAlert & Un manutentore gestisce un log di errore & Manutentore, LogErrori & stato\_ticket, data\_presa\_carico \\
\hline

ImbarcoMerce & Un container è imbarcato su un volo & Container, Volo & peso\_registrato, data\_imbarco, contenuto \\
\hline

Configurazione & Una classe passeggeri è configurata su un aereo di linea & AereoDiLinea, ClassePasseggeri & numero\_posti, spazio\_gambe \\
\hline

Dotazione & Componenti tecniche installate su un modello di aereo & TipoAereo, Componente & -- \\
\hline

AbilitazioneVolo & Un pilota è abilitato a un modello di aereo & Pilota, TipoAereo & data\_conseguimento, data\_scadenza \\
\hline

AssegnazioneGate & Una compagnia aerea ha uno o più gate assegnati & CompagniaAerea, Gate & -- \\
\hline

Possesso & Una compagnia possiede uno o più aerei & CompagniaAerea, Aereo & -- \\
\hline

CertificazioneManutenzione & Un manutentore certifica un aereo dopo un intervento & Manutentore, Aereo & data\_scadenza \\
\hline

Modello & Un aereo è associato a un modello tecnico & Aereo, TipoAereo & -- \\
\hline

Offre & Una classe passeggeri include un servizio aggiuntivo & ClassePasseggeri, ServizioAggiuntivo & -- \\
\hline

Origine & Una rotta pianificata ha un aeroporto di partenza & RottaPianificata, Aeroporto & -- \\
\hline

Destinazione & Una rotta pianificata ha un aeroporto di arrivo & RottaPianificata, Aeroporto & -- \\
\hline

Disponibilità & Una fascia oraria è disponibile in un aeroporto & FasciaOraria, Aeroporto & -- \\
\hline

\end{tabular}
\caption{Relazioni individuate nello schema concettuale, con componenti e attributi.}
\end{table}


\section{Progettazione Logica}

Nel paragrafo che segue, parleremo del processo di ristrutturazione dello schema concettuale. L’obiettivo principale è rendere la rappresentazione dei dati più efficace, eliminando i valori nulli che non servono e preparando lo schema per la traduzione nel formato relazionale. Iniziamo con un’analisi delle ridondanze nel modello, per capire quali elementi tenere e quali è meglio togliere. Dopo, ci occuperemo di rimuovere le generalizzazioni. Alla fine, presenteremo lo schema E-R ristrutturato in base a questi criteri.

\subsection{Analisi delle Ridondanze}

\subsubsection*{Ridondanza 1: \textit{numero\_voli\_giornalieri} in \texttt{Aeroporto}}
L’attributo \textit{numero\_voli\_giornalieri} nell’entità \texttt{Aeroporto} 
rappresenta il numero di voli (in partenza o in arrivo) associati ad uno specifico
aeroporto in una determinata data. Tale informazione può essere ottenuta
calcolando dinamicamente il numero di voli collegati tramite le relazioni
\texttt{Partenza(Aeroporto,Volo)} e \texttt{Arrivo(Aeroporto,Volo)}.

Tuttavia, il conteggio dei voli giornalieri comporta un costo elevato, poiché 
ogni aeroporto può avere migliaia di voli al giorno. Per questo motivo 
si valuta la convenienza di mantenere l’attributo ridondante
\textit{NumeroVoliGiornalieri} rispetto al ricalcolo continuo tramite query.

Consideriamo le seguenti operazioni tipiche:
\begin{itemize}
    \item \textbf{Operazione 1 (50 al giorno):} inserimento di un nuovo volo pianificato;
    \item \textbf{Operazione 2 (10\,000 al giorno):} consultazione del numero giornaliero di voli
    di un aeroporto.
\end{itemize}

I seguenti volumi sono considerati realistici per un sistema aeroportuale:
\begin{center}
\begin{tabular}{|c|c|c|}
\hline
Concetto & Tipo & Volume \\
\hline
Aeroporto & E & 120 \\
Volo & E & 80\,000 \\
Partenza/Arrivo & R & 160\,000 \\
\hline
\end{tabular}
\end{center}

\paragraph{Analisi con ridondanza}

L’inserimento di un nuovo volo comporta:
\begin{itemize}
    \item la creazione dell’entità \texttt{Volo};
    \item l’inserimento delle due relazioni \texttt{Partenza} e \texttt{Arrivo};
    \item la lettura e l’aggiornamento dell’attributo ridondante \texttt{numero\_voli\_giornalieri}
          nell’entità \texttt{Aeroporto}.
\end{itemize}

\begin{table}[H]
\centering
\begin{tabular}{|c|c|c|c|}
\hline
Concetto & Costrutto & Accessi & Tipo \\
\hline
Volo & E & 1 & S $\times$ 50 \\
Partenza & R & 1 & S $\times$ 50 \\
Arrivo & R & 1 & S $\times$ 50 \\
Aeroporto & E & 1 & L $\times$ 50 \\
Aeroporto & E & 1 & S $\times$ 50 \\
\hline
\end{tabular}
\caption{Operazione 1 con ridondanza: inserimento nuovo volo.}
\end{table}

Per la consultazione del numero di voli giornalieri:
\begin{table}[H]
\centering
\begin{tabular}{|c|c|c|c|}
\hline
Concetto & Costrutto & Accessi & Tipo \\
\hline
Aeroporto & E & 1 & L $\times$ 10\,000 \\
\hline
\end{tabular}
\caption{Operazione 2 con ridondanza: consultazione del numero giornaliero di voli.}
\end{table}
\[
\text{Costo giornaliero} = 50 \times 9 + 10\,000 \times 1 = 450 + 10\,000 = 10\,450.
\]

\paragraph{Analisi senza ridondanza}

In assenza dell’attributo ridondante, la consultazione giornaliera richiede 
di contare i voli connessi tramite \texttt{Partenza} e \texttt{Arrivo}.  
In media un aeroporto gestisce circa 800 voli al giorno.

L’inserimento di un nuovo volo richiede soltanto la registrazione
dell’entità \texttt{Volo} e delle relazioni \texttt{Partenza} e \texttt{Arrivo}:

\begin{table}[H]
\centering
\begin{tabular}{|c|c|c|c|}
\hline
Concetto & Costrutto & Accessi & Tipo \\
\hline
Volo & E & 1 & S $\times$ 50 \\
Partenza & R & 1 & S $\times$ 50 \\
Arrivo & R & 1 & S $\times$ 50 \\
\hline
\end{tabular}
\caption{Operazione 1 senza ridondanza.}
\end{table}

Per la consultazione:

\begin{table}[H]
\centering
\begin{tabular}{|c|c|c|c|}
\hline
Concetto & Costrutto & Accessi & Tipo \\
\hline
Aeroporto & E & 1 & L $\times$ 10\,000 \\
Partenza & R & 800 & L $\times$ 10\,000 \\
Arrivo & R & 800 & L $\times$ 10\,000 \\
\hline
\end{tabular}
\caption{Operazione 2 senza ridondanza: conteggio dei voli giornalieri.}
\end{table}
\[
\text{Costo giornaliero} = 50 \times 6 + 10\,000 \times 1\,601 = 300 + 16\,010\,000 = 16\,010\,300.
\]

\paragraph{Conclusione.}  
La consultazione quotidiana, pari a 10\,000 accessi, comporterebbe senza ridondanza
circa $16$ milioni di letture, rendendo l'operazione estremamente costosa.  
L’analisi suggerisce pertanto di \textbf{mantenere l’attributo ridondante numero\_voli\_giornalieri},
che ottimizza sensibilmente le interrogazioni frequenti a scapito di un modesto 
incremento nel costo di scrittura.

\subsubsection*{Ridondanza 2: \textit{peso\_totale\_carico} in \texttt{Volo}}

Il peso totale delle merci caricate su un volo è un valore derivabile dalla relazione
\texttt{Imbarco\_Merce}. Per ogni coppia \((c,v)\) (container–volo), tale relazione
registra il valore \textit{Peso\_Registrato}. Pertanto:

\[
\text{Peso\_Totale\_Carico}(v) = 
\sum_{\,c \in \text{Container}(v)} \text{Peso\_Registrato}(c,v)
\]

Si valuta se tenere l'attributo ridondante \textit{Peso\_Totale\_Carico}
nell'entità \texttt{Volo}, aggiornato a ogni operazione di imbarco o sbarco.

Consideriamo le operazioni tipiche:
\begin{itemize}
    \item \textbf{Operazione 1 (circa 50 al giorno):} inserimento o rimozione di un container da un volo;
    \item \textbf{Operazione 2 (circa 2\,000 al giorno):} consultazione del peso totale del carico di un volo.
\end{itemize}

Assumiamo costi di accesso: L=1, S=2.

I volumi rilevanti del sistema sono:
\begin{center}
\begin{tabular}{|c|c|c|}
\hline
Concetto & Tipo & Volume \\
\hline
Volo & E & 8\,000 \\
Container & E & 40\,000 \\
Imbarco\_Merce & R & 60\,000 \\
\hline
\end{tabular}
\end{center}

\paragraph{Con ridondanza.}
L'operazione di aggiornamento del peso totale richiede:
\begin{itemize}
    \item aggiornare \texttt{Imbarco\_Merce} (1S);
    \item leggere \texttt{Volo} (1L) e aggiornarne \textit{peso\_totale\_carico} (1S).
\end{itemize}

\begin{table}[H]
\centering
\begin{tabular}{|c|c|c|c|}
\hline
Concetto & Costrutto & Accessi & Tipo \\
\hline
Imbarco\_Merce & R & 1 & S $\times$ 50 \\
Volo & E & 1 & L $\times$ 50 \\
Volo & E & 1 & S $\times$ 50 \\
\hline
\end{tabular}
\caption{Operazione 1 con ridondanza: aggiornamento del peso totale.}
\end{table}

Per la consultazione del peso totale:
\begin{table}[H]
\centering
\begin{tabular}{|c|c|c|c|}
\hline
Concetto & Costrutto & Accessi & Tipo \\
\hline
Volo & E & 1 & L $\times$ 2\,000 \\
\hline
\end{tabular}
\caption{Operazione 2 con ridondanza: lettura diretta del valore.}
\end{table}

\[
\text{Costo giornaliero} =
50 \times (2 + 1 + 2) + 2\,000 \times 1
= 250 + 2\,000 = 2\,250.
\]

\paragraph{Senza ridondanza.}
Ogni consultazione del peso richiede di:
\begin{itemize}
    \item leggere il volo (1L);
    \item leggere tutti i container imbarcati: in media 8 per volo (8L);
    \item effettuare una somma su 20 valori.
\end{itemize}

\begin{table}[H]
\centering
\begin{tabular}{|c|c|c|c|}
\hline
Concetto & Costrutto & Accessi & Tipo \\
\hline
Imbarco\_Merce & R & 1 & S $\times$ 50 \\
\hline
\end{tabular}
\caption{Operazione 1 senza ridondanza.}
\end{table}

\begin{table}[H]
\centering
\begin{tabular}{|c|c|c|c|}
\hline
Concetto & Costrutto & Accessi & Tipo \\
\hline
Volo & E & 1 & L $\times$ 2\,000 \\
Imbarco\_Merce & R & 8 & L $\times$ 2\,000 \\
\hline
\end{tabular}
\caption{Operazione 2 senza ridondanza: scansione dei container imbarcati.}
\end{table}

Costo giornaliero senza ridondanza:
\[
50 \times 2 + 2\,000 \times (1 + 8) = 18\,100.
\]

\paragraph{Conclusione.}
Senza ridondanza, la consultazione comporta circa 42\,000 accessi al giorno, dovuti
alla scansione dei container imbarcati. Con la ridondanza il costo scende a circa
2\,250 accessi. L’analisi indica chiaramente che conviene mantenere l’attributo 
ridondante \textit{peso\_totale\_carico} nell’entità \texttt{Volo}.


\subsection{Eliminazione delle Generalizzazioni}

In questa parte si eliminano le generalizzazioni individuate 
nella fase di Progettazione Concettuale. L'obiettivo è  arrivare a uno schema che non contenga costrutti che non si possano tradurre in un modello relazionale, in modo da ridurre gli attributi che non si possono applicare e garantire una gestione corretta delle chiavi e dei vincoli di integrità.

Le generalizzazioni presenti nel nostro schema riguardano:
\begin{itemize}
    \item \textbf{Aeroporto}, suddiviso in Hub Internazionale, Regionale e Base Militare;
    \item \textbf{Aereo}, suddiviso in Aereo di Linea, Aereo Cargo, Aereo Privato e Aereo Militare;
    \item \textbf{Personale}, suddiviso in Pilota, Controllore di Volo e Manutentore;
    \item \textbf{Log di Transito}, suddiviso in Log Dati di Volo, Log Sensori GPS e Log Errori.
\end{itemize}

Le modalità con cui tali generalizzazioni vengono eliminate sono descritte qui di seguito.

\paragraph{Aeroporto}
La generalizzazione su Aeroporto è totale e disgiunta, dato che ogni aeroporto deve appartenere esclusivamente a una delle tre categorie previste. Sebbene, in teoria, sarebbe ideale accorpare gli attributi del padre nelle entità figlie, questo non è praticabile in questo contesto. L’entità Aeroporto è coinvolta in molte relazioni (Partenza, Arrivo, Presidio, Origine, Destinazione, Disponibilità), e toglierla comporterebbe la necessità di triplicare tutte queste associazioni, rendendo tutto molto più complesso.
Dunque l’entità padre Aeroporto si collega alle tre entità figlie tramite relazioni di identificazione (IS\_REGIONALE, IS\_HUB, IS\_MILITARE). Ogni aeroporto conserva gli attributi comuni nel padre, mentre quelli specifici vanno nelle rispettive entità figlie. Questa soluzione centralizza le relazioni, semplificando le query, ma richiede l’implementazione di vincoli applicativi per assicurare che ogni aeroporto appartenga a una e solo a una categoria.

\paragraph{Aereo}
La generalizzazione su Aereo è totale e sovrapposta, in quanto un aereo può avere più ruoli operativi simultaneamente (per esempio, un aereo cargo può essere utilizzato anche come velivolo militare). Proprio per questa sovrapposizione, è fondamentale mantenere l’entità padre Aereo, che contiene gli attributi comuni (matricola, data\_immatricolazione, ore\_volo\_accumulate). Le specializzazioni vengono rappresentate tramite le entità separate AereoDiLinea, AereoCargo, AereoPrivato e AereoMilitare, ciascuna collegata al padre attraverso relazioni di identificazione (IS\_LINEA, IS\_CARGO, IS\_PRIVATO, IS\_MILITARE) e attraverso la matricola (che funge da chiave primaria e chiave esterna).
Questa soluzione permette di associare un singolo aereo a più categorie contemporaneamente, inserendo la stessa matricola in più tabelle figlie, evitando duplicazioni di attributi comuni e garantendo la coerenza nei dati condivisi. Eliminare il padre porterebbe a ridondanze e complessità nella gestione di aerei multi-ruolo.

\paragraph{Personale}
La generalizzazione su Personale è parziale e disgiunta. Non tutti i membri del personale svolgono ruoli specializzati e ciascun dipendente può appartenere al massimo a una sola delle categorie previste (pilota, controllore o manutentore). Mantenendo l’entità padre Personale, che contiene attributi comuni a tutti (identificativo, nome, cognome, data di nascita, data di assunzione), si creano tre entità figlie: Pilota, ControlloreVolo e Manutentore, collegate al padre tramite relazioni di identificazione (IS\_PILOTA, IS\_CONTROLLORE, IS\_MANUTENTORE) e tramite l’identificativo.
La parzialità della gerarchia richiede il mantenimento del padre, poiché ci sono membri del personale (per esempio, personale amministrativo o di supporto) che non rientrano in nessuna specializzazione. La disgiunzione assicura che nessun dipendente possa ricoprire più ruoli specializzati contemporaneamente, mantenendo separate competenze e responsabilità operative.

\paragraph{Log di Transito}
La generalizzazione su LogTransito è totale e disgiunta. Ogni rilevazione telemetrica deve appartenere obbligatoriamente a una sola delle tre categorie: GPS, dati di volo o errori. Per questa gerarchia, si adotta un approccio di accorpamento nelle entità figlie, eliminando l’entità padre e creando tre entità indipendenti: LogGPS, LogDatiVolo e LogErrori. Ognuna di queste entità contiene sia attributi comuni (id\_volo, timestamp\_rilevazione, altitudine, velocità) sia specifici per la propria tipologia (ad esempio, latitudine e longitudine per i log GPS, o codice\_errore e severità per i log di errore).

\medskip
\noindent
Il diagramma E-R ristrutturato tiene conto di queste modifiche.


\begin{figure}[H] 
    \centering
    \includegraphics[width=\textwidth]{schema_er_ristrutturato.drawio.png}
    \caption{Diagramma Entità-Relazione Ristrutturato}
    \label{fig:schema_er}
\end{figure}

\subsection{Schema Relazionale}
Lo schema ristrutturato contiene esclusivamente costrutti direttamente 
mappabili nel modello relazionale.  
Di seguito viene riportato lo schema logico completo, dove l’asterisco (*)
denota gli attributi che possono assumere valori nulli.

\begin{itemize}

% --- AEROPORTI E LOGISTICA ---

\item \textbf{Aeroporto}(\underline{codice\_IATA}, nome, citta, nazione, coordinate, numero\_piste, numero\_voli\_giornalieri)

\item \textbf{HubInternazionale}(\underline{codice\_IATA}, numero\_terminal)  
\hspace{3mm} codice\_IATA $\rightarrow$ Aeroporto.codice\_IATA

\item \textbf{Regionale}(\underline{codice\_IATA}, lunghezza\_max\_pista, notturno)  
\hspace{3mm} codice\_IATA $\rightarrow$ Aeroporto.codice\_IATA

\item \textbf{BaseMilitare}(\underline{codice\_IATA}, codice\_nato, livello\_sicurezza)  
\hspace{3mm} codice\_IATA $\rightarrow$ Aeroporto.codice\_IATA

\item \textbf{ProtocolloSicurezza}(\underline{id\_protocollo}, nome, descrizione*)

\item \textbf{Applica}(\underline{codice\_IATA, id\_protocollo, data\_inizio})  
\hspace{3mm} codice\_IATA $\rightarrow$ BaseMilitare.codice\_IATA
\hspace{3mm} id\_protocollo $\rightarrow$ ProtocolloSicurezza.id\_protocollo

\item \textbf{Gate}(\underline{codice\_hub}, \underline{numero\_gate}, piano, tunnel)  
\hspace{3mm} codice\_hub $\rightarrow$ HubInternazionale.codice\_IATA

\item \textbf{DivietoOperativo}(\underline{codice\_IATA}, \underline{codice\_modello})
\hspace{3mm} codice\_IATA $\rightarrow$ Regionale.codice\_IATA
\hspace{3mm} codice\_modello $\rightarrow$ TipoAereo.codice\_modello

% --- TIPO AEREO E COMPONENTI ---

\item \textbf{TipoAereo}(\underline{codice\_modello}, costruttore, autonomia, velocita\_crociera)

\item \textbf{Componente}(\underline{codice\_modello}, \underline{numero\_parte}, descrizione)  
\hspace{3mm} codice\_modello $\rightarrow$ TipoAereo.codice\_modello

% --- AEREI ---

\item \textbf{Aereo}(\underline{matricola}, data\_immatricolazione, ore\_volo\_accumulate, codice\_modello)  
\hspace{3mm} codice\_modello $\rightarrow$ TipoAereo.codice\_modello

\item \textbf{AereoDiLinea}(\underline{matricola}, wifi, intrattenimento*)  
\hspace{3mm} matricola $\rightarrow$ Aereo.matricola

\item \textbf{AereoCargo}(\underline{matricola}, peso\_carico\_max)  
\hspace{3mm} matricola $\rightarrow$ Aereo.matricola

\item \textbf{AereoPrivato}(\underline{matricola}, proprietario, servizi\_inclusi*)  
\hspace{3mm} matricola $\rightarrow$ Aereo.matricola

\item \textbf{AereoMilitare}(\underline{matricola}, codice\_missione*, ente\_operativo)  
\hspace{3mm} matricola $\rightarrow$ Aereo.matricola

% --- COMPAGNIE AEREE ---

\item \textbf{CompagniaAerea}(\underline{codice}, nome, nazione, sito\_web*)

\item \textbf{Possesso}(\underline{codice\_compagnia}, \underline{matricola})  
\hspace{3mm} codice\_compagnia $\rightarrow$ CompagniaAerea.codice
\hspace{3mm} matricola $\rightarrow$ Aereo.matricola

\item \textbf{AssegnazioneGate}(\underline{codice\_compagnia}, \underline{codice\_hub}, \underline{numero\_gate})  
\hspace{3mm} codice\_compagnia $\rightarrow$ CompagniaAerea.codice
\hspace{3mm} (codice\_hub, numero\_gate) $\rightarrow$ Gate.codice\_hub, Gate.numero\_gate

% --- PERSONALE ---

\item \textbf{Personale}(\underline{id\_personale}, nome, cognome, data\_nascita, data\_assunzione)

\item \textbf{Pilota}(\underline{id\_personale}, numero\_licenza, ore\_volo, ultima\_visita)  
\hspace{3mm} id\_personale $\rightarrow$ Personale.id\_personale

\item \textbf{ControlloreVolo}(\underline{id\_personale}, livello\_certificazione, scadenza\_abilitazione)  
\hspace{3mm} id\_personale $\rightarrow$ Personale.id\_personale

\item \textbf{Manutentore}(\underline{id\_personale}, qualifica)  
\hspace{3mm} id\_personale $\rightarrow$ Personale.id\_personale

\item \textbf{AbilitazioneVolo}(\underline{id\_personale}, \underline{codice\_modello}, data\_conseguimento, data\_scadenza)  
\hspace{3mm} id\_personale $\rightarrow$ Pilota.id\_personale 
\hspace{3mm} codice\_modello $\rightarrow$ TipoAereo.codice\_modello

% --- ROTTE E SLOT ---

\item \textbf{RottaPianificata}(\underline{id\_rotta}, nome, distanza, durata\_stimata, consumo\_previsto, aeroporto\_origine, aeroporto\_destinazione)  
\hspace{3mm} aeroporto\_origine $\rightarrow$ Aeroporto.codice\_IATA
\hspace{3mm} aeroporto\_destinazione $\rightarrow$ Aeroporto.codice\_IATA

\item \textbf{FasciaOraria}(\underline{id\_slot}, codice\_IATA, ora\_inizio, ora\_fine, tipologia)  
\hspace{3mm} codice\_IATA $\rightarrow$ Aeroporto.codice\_IATA

% --- VOLI ---

\item \textbf{Volo}(\underline{id\_volo}, codice\_volo, data\_ora\_programmata, 
    data\_ora\_effettiva*, stato, peso\_totale\_carico, matricola, 
    aeroporto\_partenza, aeroporto\_arrivo, id\_rotta*, 
    slot\_decollo*, slot\_atterraggio*)  
\hspace{3mm} matricola $\rightarrow$ Aereo.matricola  
\hspace{3mm} aeroporto\_partenza $\rightarrow$ Aeroporto.codice\_IATA
\hspace{3mm} aeroporto\_arrivo $\rightarrow$ Aeroporto.codice\_IATA
\hspace{3mm} id\_rotta $\rightarrow$ RottaPianificata.id\_rotta
\hspace{3mm} slot\_decollo $\rightarrow$ FasciaOraria.id\_slot
\hspace{3mm} slot\_atterraggio $\rightarrow$ FasciaOraria.id\_slot

% --- EQUIPAGGIO E CONTROLLO ---

\item \textbf{Equipaggio}(\underline{id\_personale}, \underline{id\_volo}, ruolo, ore\_servizio)  
\hspace{3mm} id\_personale $\rightarrow$ Personale.id\_personale
\hspace{3mm} id\_volo $\rightarrow$ Volo.id\_volo

\item \textbf{Monitoraggio}(\underline{id\_personale}, \underline{id\_volo}, fase\_volo)  
\hspace{3mm} id\_personale $\rightarrow$ ControlloreVolo.id\_personale
\hspace{3mm} id\_volo $\rightarrow$ Volo.id\_volo

\item \textbf{Presidio}(\underline{id\_personale}, \underline{codice\_IATA}, \underline{data\_inizio}, data\_fine*, settore)  
\hspace{3mm} id\_personale $\rightarrow$ ControlloreVolo.id\_personale
\hspace{3mm} codice\_IATA $\rightarrow$ Aeroporto.codice\_IATA

% --- CONTAINER E MERCI ---

\item \textbf{Container}(\underline{codice\_container}, peso\_tara, tipologia)

\item \textbf{ImbarcoMerce}(\underline{id\_volo}, \underline{codice\_container}, peso\_registrato, data\_imbarco, contenuto*)  
\hspace{3mm} id\_volo $\rightarrow$ Volo.id\_volo  
\hspace{3mm} codice\_container $\rightarrow$ Container.codice\_container

% --- CLASSI PASSEGGERI ---

\item \textbf{ClassePasseggeri}(\underline{id\_classe}, nome\_commerciale, priorita\_imbarco, descrizione)

\item \textbf{Configurazione}(\underline{matricola}, \underline{id\_classe}, numero\_posti, spazio\_gambe)  
\hspace{3mm} matricola $\rightarrow$ AereoDiLinea.matricola
\hspace{3mm} id\_classe $\rightarrow$ ClassePasseggeri.id\_classe

\item \textbf{ServizioAggiuntivo}(\underline{id\_classe}, \underline{nome\_servizio}, descrizione*)  
\hspace{3mm} id\_classe $\rightarrow$ ClassePasseggeri.id\_classe

% --- LOG E MANUTENZIONE ---

\item \textbf{LogGPS}(\underline{id\_volo}, \underline{timestamp\_rilevazione}, altitudine, velocita, latitudine, longitudine)  
\hspace{3mm} id\_volo $\rightarrow$ Volo.id\_volo

\item \textbf{LogDatiVolo}(\underline{id\_volo}, \underline{timestamp\_rilevazione}, altitudine, velocita, livello\_carburante, pressione, temperatura)  
\hspace{3mm} id\_volo $\rightarrow$ Volo.id\_volo

\item \textbf{LogErrori}(\underline{id\_volo}, \underline{timestamp\_rilevazione}, altitudine, velocita, codice\_errore, messaggio*, severita)  
\hspace{3mm} id\_volo $\rightarrow$ Volo.id\_volo

\item \textbf{GestioneAlert}(\underline{id\_personale}, \underline{id\_volo}, \underline{timestamp\_rilevazione}, stato, data\_presa\_carico*)  
\hspace{3mm} id\_personale $\rightarrow$ Manutentore.id\_personale
\hspace{3mm} (id\_volo, timestamp\_rilevazione) $\rightarrow$ LogErrori(id\_volo, timestamp\_rilevazione)

\item \textbf{CertificazioneManutenzione}(\underline{id\_personale}, \underline{matricola}, \underline{data\_certificazione}, data\_scadenza)  
\hspace{3mm} id\_personale $\rightarrow$ Manutentore.id\_personale
\hspace{3mm} matricola $\rightarrow$ Aereo.matricola

\end{itemize}

\section{Database PostgreSQL e Query}

\subsection*{Query 1 – Piloti con il maggior numero di modelli di aereo abilitati}

La seguente query determina quanti modelli di aereo diversi è abilitato a pilotare
ciascun pilota. La query utilizza \texttt{GROUP BY}, funzioni di aggregazione e un join 
su tre tabelle (\texttt{Personale}, \texttt{Pilota}, \texttt{AbilitazioneVolo}).

\begin{lstlisting}[style=sqlstyle]
SELECT 
    p.id_personale,
    p.nome,
    p.cognome,
    COUNT(a.codice_modello) AS numero_modelli
FROM Personale p
JOIN Pilota pl ON p.id_personale = pl.id_personale
JOIN AbilitazioneVolo a ON pl.id_personale = a.id_personale
GROUP BY p.id_personale, p.nome, p.cognome
ORDER BY numero_modelli DESC;
\end{lstlisting}

\subsection*{Query 2 – Aeroporti con più traffico totale (arrivi + partenze)}

La query calcola il numero complessivo di partenze e arrivi per ciascun aeroporto,
valutando così il traffico totale. Utilizza join duplicati sulla tabella \texttt{Volo} e
aggregazioni.

\begin{lstlisting}[style=sqlstyle]
SELECT 
    a.codice_IATA,
    a.nome,
    COUNT(v1.id_volo) AS partenze,
    COUNT(v2.id_volo) AS arrivi,
    COUNT(v1.id_volo) + COUNT(v2.id_volo) AS traffico_totale
FROM star.Aeroporto a
LEFT JOIN star.Volo v1 ON v1.aeroporto_partenza = a.codice_IATA
LEFT JOIN star.Volo v2 ON v2.aeroporto_arrivo = a.codice_IATA
GROUP BY a.codice_IATA, a.nome
ORDER BY traffico_totale DESC;
\end{lstlisting}

\subsection*{Query 3 – Compagnie con il numero totale di aerei posseduti}

La query utilizza \texttt{GROUP BY} e l'operatore \texttt{HAVING} per selezionare
solo le compagnie che possiedono più di un aereo.

\begin{lstlisting}[style=sqlstyle]
SELECT 
    c.codice_compagnia,
    c.nome,
    COUNT(p.matricola) AS numero_aerei
FROM star.CompagniaAerea c
JOIN star.Possesso p ON c.codice_compagnia = p.codice_compagnia
GROUP BY c.codice_compagnia, c.nome
HAVING COUNT(p.matricola) > 1
ORDER BY numero_aerei DESC;
\end{lstlisting}

\subsection*{Query 4 – Velocità media reale di ogni volo con almeno due rilevazioni GPS}

Questa query calcola la velocità media dei voli utilizzando i dati registrati dai
sensori GPS. Vengono considerati solo i voli con almeno due rilevazioni.

\begin{lstlisting}[style=sqlstyle]
SELECT 
    v.id_volo,
    v.codice_volo,
    AVG(l.velocita) AS velocita_media
FROM star.Volo v
JOIN star.LogGPS l ON v.id_volo = l.id_volo
GROUP BY v.id_volo, v.codice_volo
HAVING COUNT(l.timestamp_rilevazione) >= 2
ORDER BY velocita_media DESC;
\end{lstlisting}

\subsection*{Query 5 – Aerei più utilizzati: totale ore di volo reali per aereo}

La query calcola le ore totali di volo accumulate da ogni aereo sulla base dei
log GPS (differenza tra prima e ultima rilevazione). L’aggregazione utilizza
funzioni temporali di PostgreSQL.

\begin{lstlisting}[style=sqlstyle]
SELECT
    a.matricola,
    t.codice_modello,
    EXTRACT(EPOCH FROM (MAX(l.timestamp_rilevazione) - MIN(l.timestamp_rilevazione))) / 3600
        AS ore_volo_registrate
FROM Aereo a
JOIN star.TipoAereo t ON a.codice_modello = t.codice_modello
JOIN star.LogGPS l ON l.id_volo IN (
    SELECT id_volo FROM star.Volo WHERE matricola = a.matricola
)
GROUP BY a.matricola, t.codice_modello
ORDER BY ore_volo_registrate DESC;
\end{lstlisting}

\subsection*{Indice per l'ottimizzazione della query sul traffico aeroportuale}

L’indice è stato introdotto per ottimizzare la query che calcola il traffico
totale (arrivi + partenze) per ciascun aeroporto. In questa query, gli
attributi \texttt{aeroporto\_partenza} e \texttt{aeroporto\_arrivo} della tabella
\texttt{Volo} sono al centro delle operazioni di raggruppamento e dei due join
che collegano i voli agli aeroporti.

\begin{lstlisting}[style=sqlstyle]
CREATE INDEX idx_volo_partenza ON Volo(aeroporto_partenza);
CREATE INDEX idx_volo_arrivo   ON Volo(aeroporto_arrivo);
\end{lstlisting}

La scelta di due indici dedicati consente a PostgreSQL di ridurre
significativamente il costo dei join, evitando una doppia scansione completa
della tabella \texttt{Volo}. In presenza degli indici, il planner può infatti
eseguire ricerche mirate sulle tuple rilevanti e calcolare le aggregazioni
(\texttt{COUNT}) quasi interamente tramite index scan, minimizzando l’accesso
alla tabella principale. Ciò garantisce un miglioramento sostanziale nelle
query analitiche sul traffico aeroportuale.





\end{document}