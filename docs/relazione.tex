\pdfminorversion=4
\documentclass[]{article}
\usepackage[utf8]{inputenc}
\usepackage{amssymb,latexsym,amsmath}
\usepackage[a4paper,top=3cm,bottom=2cm,left=3cm,right=3cm,marginparwidth=1.75cm]{geometry}
\usepackage{graphicx}
\usepackage{float}
\usepackage[colorlinks=true, allcolors=blue]{hyperref}
\begin{document}

Il progetto \textbf{STAR} (Sistema di Tracciamento dello Spazio Aereo e Rotte) è una banca dati pensata per gestire in modo integrato e monitorare il traffico aereo sia civile che militare. Questo sistema segue tutte le fasi operative di un volo, dalla pianificazione della rotta e prenotazione degli slot negli aeroporti, fino alla storicizzazione delle operazioni effettuate e alla gestione dei dati telemetrici


La banca dati si concentra su tre aree principali:

\begin{itemize}
    \item \textbf{Infrastruttura e Logistica}: Qui si gestiscono gli aeroporti, sia hub che regionali e militari, i gate e tutta la complessa logistica di carico merce (container) e passeggeri.
    
    \item \textbf{Flotta e Risorse Umane}: Questo aspetto si occupa in dettaglio degli aeromobili, con tutte le loro configurazioni, e del personale specializzato come piloti, controllori e manutentori, tenendo traccia delle loro abilitazioni, turni e certificazioni.

    
    \item \textbf{Sicurezza e Monitoraggio}: Archiviazione dei log operativi (rilevazioni GPS campionate, parametri tecnici e messaggi di errore) finalizzata alla manutenzione predittiva e alla gestione dei protocolli di sicurezza per le aree militari.
\end{itemize}

L'obiettivo è quello di supportare le decisioni per ottimizzare le risorse aeroportuali e garantire una completa tracciabilità degli eventi di volo e di manutenzione.


\section{Analisi dei Requisiti}

Questa sezione riassume i requisiti informativi richiesti dal sistema da sviluppare. 
L'obiettivo è descrivere i concetti principali, le loro caratteristiche rilevanti e 
le relazioni che intercorrono tra essi.

\subsection*{Aeroporti}
Ogni aeroporto è identificato da un codice internazionale e contiene le seguenti informazioni:
\begin{itemize}
    \item codice IATA;
    \item nome;
    \item città e nazione in cui si trova;
    \item coordinate GPS;
    \item numero di piste disponibili.
    \item numero di voli giornalieri.
\end{itemize}
Si tracciano il settore assegnato, data e ora di inizio, e data e ora di fine dei controllori di volo che presidiano un aereoporto.
Gli aeroporti si suddividono in tre categorie:
\begin{itemize}
    \item \textbf{Hub Internazionale}, caratterizzato dal numero di terminal presenti;
    \item \textbf{Regionale}, che presenta limitazioni operative, quali la lunghezza massima della pista e l'eventuale possibilità di operare in orario notturno;
    \item \textbf{Base Militare}, con codice NATO e livello minimo di sicurezza, oltre all'applicazione di specifici protocolli di sicurezza.
\end{itemize}
Un Hub Internazionale può inoltre possedere diversi gate, che rappresentano i punti fisici di imbarco, mentre una Base Militare applica protocolli di sicurezza

\subsection*{Aerei}
Ogni aereo è identificato da una matricola univoca e contiene:
\begin{itemize}
    \item data di immatricolazione;
    \item ore di volo accumulate.
\end{itemize}
Gli aerei si suddividono in:
\begin{itemize}
    \item \textbf{Aereo di Linea}, dotato di rete Wi-Fi e configurazioni di intrattenimento dedicate al trasporto passeggeri;
    \item \textbf{Aereo Cargo}, caratterizzato dal peso massimo di carico supportato;
    \item \textbf{Aereo Privato}, gestito per conto di un proprietario specifico, con eventuali servizi dedicati;
    \item \textbf{Aereo Militare}, utilizzato in operazioni governative, con un codice missione e un ente operativo.
\end{itemize}
Un aereo può svolgere contemporaneamente più ruoli (ad esempio: cargo privato, cargo militare).

\subsection*{Personale}
Il personale rappresenta tutte le persone che operano nel contesto aeroportuale. 
Per ciascun dipendente sono registrati:
\begin{itemize}
    \item identificativo univoco;
    \item nome e cognome;
    \item data di nascita;
    \item data di assunzione.
\end{itemize}
Il personale appartiene a una gerarchia parziale e disgiunta:
\begin{itemize}
    \item \textbf{Pilota}, con numero di licenza, ore di volo e data dell'ultima visita medica, possiede un abilitazione al volo per uno o piu aerei, la quale ha una data di conseguimento e una di scadenza.
    \item \textbf{Controllore di Volo}, con livello e scadenza delle certificazioni abilitative;
    \item \textbf{Manutentore}, con qualifica tecnica e responsabilità di interventi e certificazioni (si traccia la data di scadenza della certificazione).
    Gestisce anche gli alert generati dai Log Errori, i quali hanno uno stato e una data di presa in carico.
\end{itemize}

\subsection*{Voli}
Il volo rappresenta il concetto centrale del sistema. Ogni volo è identificato da un id univoco e registra:
\begin{itemize}
    \item codice del volo;
    \item data e ora programmata di partenza;
    \item data e ora effettiva di partenza;
    \item stato operativo (programmato, in corso, atterrato, cancellato);
    \item peso totale del carico delle merci.
\end{itemize}
Il personale di bordo è gestito da un \textit{Equipaggio} (si tracciano il ruolo e le ore di servizio), mentre i controllori di volo supervisionano le fasi di un volo, tracciando quale fase si ha monitorato.  
Ogni volo può prenotare specifiche fasce orarie per il decollo e l'atterraggio, 
tracciare i propri dati di telemetria tramite log dedicati e viene associato a una 
rotta pianificata. Un aereo esegue uno specifico volo.

\subsection*{Log di Transito}
I log di transito rappresentano i dati telemetrici generati durante un volo.  
Ogni rilevazione è associata a un singolo volo e a un istante temporale, e contiene 
informazioni quali altitudine e velocità.  
La gerarchia dei log è:
\begin{itemize}
    \item \textbf{Log Sensore GPS}, contenente latitudine e longitudine;
    \item \textbf{Log Dati di Volo}, con livello carburante, temperatura e pressione;
    \item \textbf{Log Errori}, con codice errore, messaggio e livello di severità.
\end{itemize}

\subsection*{Fascia Oraria}
Le fasce orarie rappresentano gli slot temporali gestiti dai singoli aereoporti destinati alle operazioni di decollo e atterraggio.  
Ogni fascia oraria contiene:
\begin{itemize}
    \item identificativo dello slot;
    \item orario di inizio e fine;
    \item tipologia dello slot.
\end{itemize}
I voli possono prenotare fasce di decollo e fasce di atterraggio distinte.

\subsection*{Rotta Pianificata}
Una rotta pianificata descrive un percorso standard e contiene:
\begin{itemize}
    \item identificativo della rotta;
    \item nome descrittivo;
    \item durata stimata;
    \item distanza;
    \item consumo previsto di carburante.
\end{itemize}
Ogni volo segue una specifica rotta.

\subsection*{Container}
I container rappresentano le unità di carico utilizzate per il trasporto merci.  
Per ciascun container sono registrati:
\begin{itemize}
    \item codice identificativo;
    \item peso della tara;
    \item tipologia.
\end{itemize}
Sono utilizzati sia per il trasporto tramite aerei cargo, dove si traccia la posizione in stiva e la data di imbarco, sia per la registrazione delle merci imbarcate su un volo, dove invece si monitorano il peso registrato durante quel volo, la data di imbarco e il contenuto.

\subsection*{Classe Passeggeri e Servizi}
La classe passeggeri rappresenta una categoria di servizio a bordo degli aerei di linea, la quale possiede un numero di posti a sedere differente, 
caratterizzata da:
\begin{itemize}
    \item identificativo della classe;
    \item spazio per le gambe.
\end{itemize}
A ciascuna classe sono associati eventuali \textit{Servizi Aggiuntivi}, ciascuno identificato da:
\begin{itemize}
    \item nome del servizio;
    \item classe a cui appartiene.
    \item descrizione.
\end{itemize}

\subsection*{Tipo di Aereo e Componenti}
Il tipo di aereo (modello) fornisce informazioni tecniche:
\begin{itemize}
    \item codice del modello;
    \item costruttore;
    \item autonomia;
    \item velocità di crociera.
\end{itemize}
Ogni modello può essere dotato di componenti tecnici specifici, 
ciascuno caratterizzato da un numero di parte e una descrizione.

\subsection*{Compagnia Aerea}
Le compagnie aeree operano negli aeroporti e sono descritte da:
\begin{itemize}
    \item codice identificativo;
    \item nome;
    \item nazionalità;
    \item sito web.
\end{itemize}
Una compagnia può possedere diversi aerei e avere uno o più gate assegnati.

\subsection*{Gate}
I gate rappresentano i punti fisici di imbarco e sbarco presenti esclusivamente negli 
\textit{Hub Internazionali}. 
La loro esistenza dipende dall'hub cui appartengono.

Ogni gate registra:
\begin{itemize}
    \item numero del gate (univoco solo all'interno dello stesso hub);
    \item piano in cui si trova il gate;
    \item presenza o meno del tunnel di imbarco (jetbridge).
\end{itemize}

Alcuni gate possono essere assegnati a specifiche compagnie aeree per l’utilizzo operativo.



\subsection*{Protocollo di Sicurezza}
I protocolli di sicurezza si applicano alle basi militari e rappresentano le procedure operative 
obbligatorie da seguire. Ogni protocollo è caratterizzato da:
\begin{itemize}
    \item identificativo del protocollo;
    \item nome;
    \item descrizione.
\end{itemize}
Le basi militari possono applicare uno o più protocolli, ciascuno con una data di inizio validità.

\subsection*{Componente}
I componenti rappresentano gli elementi tecnici standard installabili su uno specifico modello di aereo, e sono identificati solo in relazione al tipo di aereo.

Ogni componente registra:
\begin{itemize}
    \item numero di parte (univoco all'interno del modello);
    \item descrizione tecnica;
\end{itemize}
I componenti definiscono la dotazione tecnica minima associata a ogni modello di aereo.



\begin{figure}[H] 
    \centering
    \includegraphics[width=\textwidth]{schema_er.drawio.png}
    \caption{Diagramma Entità-Relazione del progetto STAR}
    \label{fig:schema_er}
\end{figure}


La Figura 1 presenta il diagramma Entità-Relazione (E-R) che riassume i concetti emersi dall'analisi dei requisiti e le loro connessioni. Il sistema rappresenta una rete aeroportuale suddivisa in tre categorie di aereoporti: Hub Internazionali, Aeroporti Regionali e Basi Militari. Questa classificazione forma una gerarchia totale e disgiunta, quindi ogni aeroporto deve appartenere, in modo esclusivo, a una di queste categorie. Gli Hub Internazionali comprendono anche diversi Gate, che sono modellati come entità deboli, identificati dalla coppia (CodAeroporto, NumeroGate). Un altro elemento chiave del dominio riguarda gli Aerei, le cui specifiche tecniche derivano dall'entità Tipo Aereo, che include il catalogo dei modelli registrati nel sistema. Gli aerei sono organizzati tramite una generalizzazione totale e sovrapposta, differenziando tra Aerei di Linea, Aerei Cargo, Aerei Privati e Aerei Militari. Questa scelta di modellazione riflette che un singolo velivolo può operare in più ruoli contemporaneamente, come nel caso di configurazioni combinate passeggeri-cargo o di freighter utilizzati in contesti militari. Gli Aerei di Linea hanno anche una Configurazione interna, definita rispetto all'entità Classe Passeggeri, che rappresenta le varie categorie di servizio disponibili a bordo (per esempio, Economy o Business). A ciascuna classe possono essere associati uno o più Servizi Aggiuntivi, modellati come entità deboli identificate dalla coppia (NomeServizio, ID Classe). La gestione operativa dei voli è affidata all'entità Volo, che rappresenta l'evento centrale del sistema. Le relazioni di Partenza e Arrivo collegano ogni volo rispettivamente all'aeroporto di partenza e quella di destinazione. Ogni volo è effettuato da un singolo aereo tramite la relazione Esegue ed è monitorato da uno o più Controllori di Volo attraverso la relazione Monitoraggio. L'entità Personale descrive le risorse umane coinvolte nelle operazioni aeroportuali ed è organizzata in una generalizzazione parziale e disgiunta nelle sottoclassi Pilota, Controllore di Volo e Manutentore. Durante l'esecuzione di un volo, vengono registrati i Log di Transito, modellati come entità deboli identificate dalla coppia (IDVolo, Timestamp). La gerarchia è totale e disgiunta e include tre tipi di log: Log Sensore GPS, Log Dati di Volo e Log Errori. I log d'errore possono generare notifiche per i manutentori tramite la relazione Gestione Alert. Per quanto riguarda la gestione delle merci, l'entità Container rappresenta le unità di carico trasportate nel sistema. La relazione Imbarco Merce associa ogni container ai voli sui quali viene caricato, registrando peso e contenuto dell'operazione. La relazione Trasporto consente di ricostruire la cronologia dei container movimentati da ciascun aereo cargo, indipendentemente dai voli. L'entità Rotta Pianificata descrive i collegamenti standard tra aeroporti e viene associata ai voli tramite la relazione Assegnazione Rotta. La Tabella 1, che trovi nella sezione successiva, riassume tutte le entità e relazioni illustrate nello schema E-R, indicando attributi e identificatori.  

\subsection*{Vincoli non rappresentabili nello schema E-R}

Alcune regole del dominio non possono essere rappresentate direttamente nel diagramma
E--R, ma costituiscono comunque vincoli fondamentali del sistema. Tra i principali:

\begin{itemize}

    \item \textbf{Un aereo classificato come Militare non può essere posseduto da una compagnia aerea commerciale}.  
    Lo schema consente la relazione \emph{Possesso} con qualsiasi aereo, ma il dominio richiede il vincolo:
    \[
        a \in \text{Aereo Militare} \;\Rightarrow\; \neg \exists c \; \text{Possesso}(c,a).
    \]

    \item \textbf{Un container può essere imbarcato su un volo solo se il peso registrato non supera la capacità di carico dell’aereo che esegue quel volo}.  
    Questo vincolo incrocia attributi e relazioni diverse (\emph{ImbarcoMerce}, \emph{Esegue}, \emph{AereoCargo}) ed è quindi non rappresentabile graficamente:
    \[
        \text{Imbarco Merce}(c,v) \;\Rightarrow\; 
        \text{peso\_registrato}(c,v) \leq \text{peso\_carico\_max}(\text{Esegue}(v)).
    \]

    \item \textbf{Un controllore di volo può monitorare un volo solo se è assegnato a un aeroporto coerente con quel volo}.  
    In particolare, se un controllore monitora un volo, deve presidiare almeno uno degli aeroporti di partenza o arrivo:
    \[
        \text{Monitoraggio}(p,v) \;\Rightarrow\; 
        \exists a \big( \text{Presidio}(p,a) 
        \wedge (a = \text{Partenza}(v) \;\vee\; a = \text{Arrivo}(v)) \big).
    \]

    \item \textbf{La gestione degli errori dipende dallo stato del ticket}.  
    In particolare, uno stato diverso da ``non assegnato'' richiede che esista almeno un manutentore associato tramite \emph{Gestione~Alert}.  
    Il legame tra stato del log e presenza di un manutentore non è esprimibile nel solo schema E--R.

    \item \textbf{Un membro dell’equipaggio può operare come Pilota solo se è abilitato al modello dell’aereo che esegue il volo}.  
    Questo vincolo incrocia \emph{Equipaggio}, \emph{Pilota}, \emph{AbilitazioneVolo}, \emph{Esegue} e \emph{TipoAereo}:
    \[
        \text{Equipaggio}(p,v) \wedge \text{Ruolo}(p,v)=\text{``Pilota''}
        \;\Rightarrow\;
        \exists t\big(
            \text{Abilitazione Volo}(p,t)
            \wedge
            t = \text{Tipo Aereo}(\text{Esegue}(v))
        \big).
    \]

\end{itemize}

\begin{table}[H]
\centering
\begin{tabular}{|p{4cm}|p{4cm}|p{4cm}|p{4cm}|}
\hline
\textbf{Entità} & \textbf{Descrizione} & \textbf{Attributi} & \textbf{Identificatore} \\
\hline

Aeroporto & Struttura aeroportuale & codice, nome, città, nazione, coordinate, numero\_piste, numero\_voli\_giornalieri& codice\\
\hline

HubInternazionale & Tipo di aeroporto & numero\_terminal & \\
\hline

AeroportoRegionale & Tipo di aeroporto & lunghezza\_pista\_max, notturno& \\
\hline

BaseMilitare & Aeroporto militare & codice\_nato, livello\_sicurezza& \\
\hline

Gate& Punto di imbarco & numero, piano, tunnel\_imbarco& (codice, numero)\\
\hline

ProtocolloSicurezza & Procedure militari & id, nome, descrizione& id\\
\hline

Aereo & Velivolo & matricola, data\_immatricolazione, ore\_volo\_accumulate& matricola\\
\hline

AereoDiLinea & Sottotipo aereo & wifi, intrattenimento& \\
\hline

AereoCargo & Sottotipo aereo & volume\_carico\_max& \\
\hline

AereoPrivato & Sottotipo aereo & nome\_proprietario, servizio& \\
\hline

AereoMilitare & Sottotipo aereo & codice\_missione, ente\_operativo& \\
\hline

TipoAereo & Catalogo modelli & codice\_modello, costruttore, autonomia, velocita\_crociera& codice\_modello\\
\hline

Componente& Componenti tecnici standard & numero, descrizione & (codice\_modello, numero)\\
\hline

Personale & Risorse umane & id, nome, cognome, data\_nascita, data\_assunzione& id\\
\hline

Pilota & Sottoclasse personale & numero\_licenza, ore\_volo\_totali, ultima\_visita\_medica& \\
\hline

ControlloreVolo & Sottoclasse personale & livello\_certificazione, scadenza\_abilitazione& \\
\hline

Manutentore & Sottoclasse personale & qualifica& \\
\hline

Volo & Evento operativo & id\_volo, codice\_volo, data\_programmata, data\_partenza\_effettiva, stato, peso\_totale\_carico& id\_volo\\
\hline

LogTransito& Rilevazioni telemetriche & timestamp, altitudine, velocità& (id\_volo, timestamp)\\
\hline

LogGPS & Sottotipo log & latitudine, longitudine& \\
\hline

LogDatiVolo & Sottotipo log & livello\_carburante, temperatura\_esterna, pressione\_cabina& \\
\hline

LogErrori & Sottotipo log & codice\_errore, messaggio, livello\_severità& \\
\hline

FasciaOraria & Slot aeroportuali & id\_slot, orario\_inizio, orario\_fine, tipo& id\_slot\\
\hline

RottaPianificata & Percorso standard & id\_rotta, nome\_rotta, consumo\_carburante, distanza, durata\_stimata& id\_rotta\\
\hline

Container & Unità di carico & codice\_container, peso\_tara, tipologia& codice\_container\\
\hline

ClassePasseggeri & Categoria servizio & id\_classe, spazio\_gambe& id\_classe\\
\hline

ServizioAggiuntivo& Servizi associati alle classi & nome\_servizio, descrizione, classe\_appartenenza& (nome\_servizio, id\_classe)\\
\hline

CompagniaAerea & Operatore aeroportuale & codice, nome, nazione, sito\_web& codice\\
\hline

\end{tabular}
\caption{Entità individuate dall’analisi dei requisiti, con attributi e identificatori.}
\end{table}

\begin{table}[H]
\centering
\begin{tabular}{|p{4cm}|p{4cm}|p{4cm}|p{4cm}|}
\hline
\textbf{Relazione} & \textbf{Descrizione} & \textbf{Componente} & \textbf{Attributi} \\
\hline

Possiede & Un Hub Internazionale possiede uno o più gate & Hub internazionale, Gate & -- \\
\hline

Applica & Una base militare applica protocolli di sicurezza & BaseMilitare, ProtocolloSicurezza & data\_inizio\_validità \\
\hline

DivietoOperativo & Un aeroporto regionale vieta certi tipi di aereo & Regionale, TipoAereo & -- \\
\hline

Esecuzione & Un aereo esegue un volo & Aereo, Volo & -- \\
\hline

Partenza & Un volo parte da un aeroporto & Volo, Aeroporto & -- \\
\hline

Arrivo & Un volo arriva in un aeroporto & Volo, Aeroporto & -- \\
\hline

AssegnazioneRotta & Una rotta viene associata a un volo & RottaPianificata, Volo & -- \\
\hline

PrenotazioneDecollo & Un volo prenota uno slot di decollo & Volo, FasciaOraria & -- \\
\hline

PrenotazioneAtterraggio & Un volo prenota uno slot di atterraggio & Volo, FasciaOraria & -- \\
\hline

Equipaggio & Piloti e assistenti operano su un volo & Personale, Volo & ruolo, ore\_servizio \\
\hline

Monitoraggio & Un controllore supervisiona un volo & ControlloreVolo, Volo & fase\_volo \\
\hline

Presidio & Un controllore è assegnato a un aeroporto & ControlloreVolo, Aeroporto & settore, data\_fine, data\_inizio \\
\hline

Traccia & Un volo genera log di transito & Volo, LogTransito & -- \\
\hline

GestioneAlert & Un manutentore gestisce un log di errore & Manutentore, LogErrori & stato\_ticket, data\_presa\_carico \\
\hline

ImbarcoMerce & Un container è imbarcato su un volo & Container, Volo & peso\_registrato, data\_imbarco, contenuto \\
\hline

Trasporto & Uno storico dei container trasportati da un aereo cargo & AereoCargo, Container & data\_carico, posizione\_stiva \\
\hline

Configurazione & Una classe passeggeri è configurata su un aereo di linea & AereoDiLinea, ClassePasseggeri & numero\_posti \\
\hline

Dotazione & Componenti tecniche installate su un modello di aereo & TipoAereo, Componente & -- \\
\hline

AbilitazioneVolo & Un pilota è abilitato a un modello di aereo & Pilota, TipoAereo & data\_conseguimento, data\_scadenza \\
\hline

AssegnazioneGate & Una compagnia aerea ha uno o più gate assegnati & CompagniaAerea, Gate & -- \\
\hline

Possesso & Una compagnia possiede uno o più aerei & CompagniaAerea, Aereo & -- \\
\hline

CertificazioneManutenzione & Un manutentore certifica un aereo dopo un intervento & Manutentore, Aereo & data\_scadenza \\
\hline

Modello & Un aereo è associato a un modello tecnico & Aereo, TipoAereo & -- \\
\hline

Offre & Una classe passeggeri include un servizio aggiuntivo & ClassePasseggeri, ServizioAggiuntivo & -- \\
\hline

Origine & Una rotta pianificata ha un aeroporto di partenza & RottaPianificata, Aeroporto & -- \\
\hline

Destinazione & Una rotta pianificata ha un aeroporto di arrivo & RottaPianificata, Aeroporto & -- \\
\hline

Disponibilità & Una fascia oraria è disponibile in un aeroporto & FasciaOraria, Aeroporto & -- \\
\hline

\end{tabular}
\caption{Relazioni individuate nello schema concettuale, con componenti e attributi.}
\end{table}


\section{Progettazione Logica}

Nel paragrafo che segue, parleremo del processo di ristrutturazione dello schema concettuale. L’obiettivo principale è rendere la rappresentazione dei dati più efficace, eliminando i valori nulli che non servono e preparando lo schema per la traduzione nel formato relazionale. Iniziamo con un’analisi delle ridondanze nel modello, per capire quali elementi tenere e quali è meglio togliere. Dopo, ci occuperemo di rimuovere le generalizzazioni. Alla fine, presenteremo lo schema E-R ristrutturato in base a questi criteri.

\subsection{Analisi delle Ridondanze}

\subsubsection*{Ridondanza 1: \textit{numero\_voli\_giornalieri} in \texttt{Aeroporto}}
L’attributo \textit{numero\_voli\_giornalieri} nell’entità \texttt{Aeroporto} 
rappresenta il numero di voli (in partenza o in arrivo) associati ad uno specifico
aeroporto in una determinata data. Tale informazione può essere ottenuta
calcolando dinamicamente il numero di voli collegati tramite le relazioni
\texttt{Partenza(Aeroporto,Volo)} e \texttt{Arrivo(Aeroporto,Volo)}.

Tuttavia, il conteggio dei voli giornalieri comporta un costo elevato, poiché 
ogni aeroporto può avere migliaia di voli al giorno. Per questo motivo 
si valuta la convenienza di mantenere l’attributo ridondante
\textit{NumeroVoliGiornalieri} rispetto al ricalcolo continuo tramite query.

Consideriamo le seguenti operazioni tipiche:
\begin{itemize}
    \item \textbf{Operazione 1 (50 al giorno):} inserimento di un nuovo volo pianificato;
    \item \textbf{Operazione 2 (10\,000 al giorno):} consultazione del numero giornaliero di voli
    di un aeroporto.
\end{itemize}

I seguenti volumi sono considerati realistici per un sistema aeroportuale:
\begin{center}
\begin{tabular}{|c|c|c|}
\hline
Concetto & Tipo & Volume \\
\hline
Aeroporto & E & 120 \\
Volo & E & 80\,000 \\
Partenza/Arrivo & R & 160\,000 \\
\hline
\end{tabular}
\end{center}

\paragraph{Analisi con ridondanza}

L’inserimento di un nuovo volo comporta:
\begin{itemize}
    \item la creazione dell’entità \texttt{Volo};
    \item l’inserimento delle due relazioni \texttt{Partenza} e \texttt{Arrivo};
    \item la lettura e l’aggiornamento dell’attributo ridondante \texttt{numero\_voli\_giornalieri}
          nell’entità \texttt{Aeroporto}.
\end{itemize}

\begin{table}[H]
\centering
\begin{tabular}{|c|c|c|c|}
\hline
Concetto & Costrutto & Accessi & Tipo \\
\hline
Volo & E & 1 & S $\times$ 50 \\
Partenza & R & 1 & S $\times$ 50 \\
Arrivo & R & 1 & S $\times$ 50 \\
Aeroporto & E & 1 & L $\times$ 50 \\
Aeroporto & E & 1 & S $\times$ 50 \\
\hline
\end{tabular}
\caption{Operazione 1 con ridondanza: inserimento nuovo volo.}
\end{table}

Per la consultazione del numero di voli giornalieri:
\begin{table}[H]
\centering
\begin{tabular}{|c|c|c|c|}
\hline
Concetto & Costrutto & Accessi & Tipo \\
\hline
Aeroporto & E & 1 & L $\times$ 10\,000 \\
\hline
\end{tabular}
\caption{Operazione 2 con ridondanza: consultazione del numero giornaliero di voli.}
\end{table}

\paragraph{Analisi senza ridondanza}

In assenza dell’attributo ridondante, la consultazione giornaliera richiede 
di contare i voli connessi tramite \texttt{Partenza} e \texttt{Arrivo}.  
In media un aeroporto gestisce circa 800 voli al giorno.

L’inserimento di un nuovo volo richiede soltanto la registrazione
dell’entità \texttt{Volo} e delle relazioni \texttt{Partenza} e \texttt{Arrivo}:

\begin{table}[H]
\centering
\begin{tabular}{|c|c|c|c|}
\hline
Concetto & Costrutto & Accessi & Tipo \\
\hline
Volo & E & 1 & S $\times$ 50 \\
Partenza & R & 1 & S $\times$ 50 \\
Arrivo & R & 1 & S $\times$ 50 \\
\hline
\end{tabular}
\caption{Operazione 1 senza ridondanza.}
\end{table}

Per la consultazione:

\begin{table}[H]
\centering
\begin{tabular}{|c|c|c|c|}
\hline
Concetto & Costrutto & Accessi & Tipo \\
\hline
Aeroporto & E & 1 & L $\times$ 10\,000 \\
Partenza & R & 800 & L $\times$ 10\,000 \\
Arrivo & R & 800 & L $\times$ 10\,000 \\
\hline
\end{tabular}
\caption{Operazione 2 senza ridondanza: conteggio dei voli giornalieri.}
\end{table}

\paragraph{Conclusione.}  
La consultazione quotidiana, pari a 10\,000 accessi, comporterebbe senza ridondanza
circa $16$ milioni di letture, rendendo l'operazione estremamente costosa.  
L’analisi suggerisce pertanto di \textbf{mantenere l’attributo ridondante numero\_voli\_giornalieri},
che ottimizza sensibilmente le interrogazioni frequenti a scapito di un modesto 
incremento nel costo di scrittura.

\subsubsection*{Ridondanza 2: \textit{peso\_totale\_carico} in \texttt{Volo}}

Il peso totale delle merci caricate su un volo è un valore derivabile dalla relazione
\texttt{Imbarco\_Merce}. Per ogni coppia \((c,v)\) (container–volo), tale relazione
registra il valore \textit{Peso\_Registrato}. Pertanto:

\[
\text{Peso\_Totale\_Carico}(v) = 
\sum_{\,c \in \text{Container}(v)} \text{Peso\_Registrato}(c,v)
\]

Si valuta se tenere l'attributo ridondante \textit{Peso\_Totale\_Carico}
nell'entità \texttt{Volo}, aggiornato a ogni operazione di imbarco o sbarco.

Consideriamo le operazioni tipiche:
\begin{itemize}
    \item \textbf{Operazione 1 (circa 50 al giorno):} inserimento o rimozione di un container da un volo;
    \item \textbf{Operazione 2 (circa 2\,000 al giorno):} consultazione del peso totale del carico di un volo.
\end{itemize}

Assumiamo costi di accesso: L=1, S=2.

I volumi rilevanti del sistema sono:
\begin{center}
\begin{tabular}{|c|c|c|}
\hline
Concetto & Tipo & Volume \\
\hline
Volo & E & 8\,000 \\
Container & E & 40\,000 \\
Imbarco\_Merce & R & 60\,000 \\
\hline
\end{tabular}
\end{center}

\paragraph{Senza ridondanza.}
Ogni consultazione del peso richiede di:
\begin{itemize}
    \item leggere il volo (1L);
    \item leggere tutti i container imbarcati: in media 20 per volo (20L);
    \item effettuare una somma su 20 valori.
\end{itemize}

\begin{table}[H]
\centering
\begin{tabular}{|c|c|c|c|}
\hline
Concetto & Costrutto & Accessi & Tipo \\
\hline
Imbarco\_Merce & R & 1 & S $\times$ 50 \\
\hline
\end{tabular}
\caption{Operazione 1 senza ridondanza.}
\end{table}

\begin{table}[H]
\centering
\begin{tabular}{|c|c|c|c|}
\hline
Concetto & Costrutto & Accessi & Tipo \\
\hline
Volo & E & 1 & L $\times$ 2\,000 \\
Imbarco\_Merce & R & 20 & L $\times$ 2\,000 \\
\hline
\end{tabular}
\caption{Operazione 2 senza ridondanza: scansione dei container imbarcati.}
\end{table}

Costo giornaliero senza ridondanza:
\[
2\,000 \times (1 + 20) = 42\,000.
\]

\paragraph{Con ridondanza.}
L'operazione di aggiornamento del peso totale richiede:
\begin{itemize}
    \item aggiornare \texttt{Imbarco\_Merce} (1S);
    \item leggere \texttt{Volo} (1L) e aggiornarne \textit{peso\_totale\_carico} (1S).
\end{itemize}

\begin{table}[H]
\centering
\begin{tabular}{|c|c|c|c|}
\hline
Concetto & Costrutto & Accessi & Tipo \\
\hline
Imbarco\_Merce & R & 1 & S $\times$ 50 \\
Volo & E & 1 & L $\times$ 50 \\
Volo & E & 1 & S $\times$ 50 \\
\hline
\end{tabular}
\caption{Operazione 1 con ridondanza: aggiornamento del peso totale.}
\end{table}

Per la consultazione del peso totale:
\begin{table}[H]
\centering
\begin{tabular}{|c|c|c|c|}
\hline
Concetto & Costrutto & Accessi & Tipo \\
\hline
Volo & E & 1 & L $\times$ 2\,000 \\
\hline
\end{tabular}
\caption{Operazione 2 con ridondanza: lettura diretta del valore.}
\end{table}

\[
\text{Costo giornaliero} =
50 \times (2 + 1 + 2) + 2\,000 \times 1
= 250 + 2\,000 = 2\,250.
\]

\paragraph{Conclusione.}
Senza ridondanza, la consultazione comporta circa 42\,000 accessi al giorno, dovuti
alla scansione dei container imbarcati. Con la ridondanza il costo scende a circa
2\,150 accessi. L’analisi indica chiaramente che conviene mantenere l’attributo 
ridondante \textit{peso\_totale\_carico} nell’entità \texttt{Volo}.


\subsection{Eliminazione delle Generalizzazioni}

In questa sezione si eliminano le generalizzazioni identificate 
nella Progettazione Concettuale. L'obiettivo è quello di arrivare a uno schema che non contenga costrutti che non si possano tradurre in un modello relazionale. Togliere le generalizzazioni aiuta a ridurre gli attributi che non si possono applicare e assicura una gestione corretta delle chiavi e dei vincoli di integrità.

Le generalizzazioni presenti nel nostro schema riguardano:
\begin{itemize}
    \item \textbf{Aeroporto}, suddiviso in Hub Internazionale, Aeroporto Regionale e Base Militare;
    \item \textbf{Aereo}, suddiviso in Aereo di Linea, Aereo Cargo, Aereo Privato e Aereo Militare;
    \item \textbf{Personale}, suddiviso in Pilota, Controllore di Volo e Manutentore;
    \item \textbf{Log di Transito}, suddiviso in Log Dati di Volo, Log Sensori GPS e Log Errori.
\end{itemize}

Le modalità con cui tali generalizzazioni vengono eliminate sono descritte qui di seguito.

\paragraph{Aeroporto}
La generalizzazione su \textbf{Aeroporto} è \emph{totale e disgiunta}, 
dato che ogni aeroporto può rientrare solo in una specifica categoria. La generalizzazione viene eliminata creando tre relazioni distinte:
\texttt{IS\_HUB}, \texttt{IS\_REG}, \texttt{IS\_MIL}.  
In queste relazioni ogni occorrenza è identificata tramite la stessa chiave primaria 
dell’entita \texttt{Aeroporto}.  
Questa trasformazione evita la presenza di attributi nulli: ad esempio, 
gli aeroporti regionali non devono contenere attributi relativi ai protocolli di sicurezza militari, e le basi militari non devono registrare il numero di terminal tipici degli hub internazionali.

\paragraph{Aereo}
La generalizzazione su \textbf{Aereo} è \emph{totale e sovrapposta}, 
poiche un aeromobile può svolgere più ruoli operativi (ad esempio, 
aereo cargo impiegato temporaneamente come velivolo militare).  
Poprio a causa della natura sovrapposta, non è possibile rimuovere l’entità padre 
senza duplicare la chiave nelle entità figlie, il che porterebbe a introdurre incoerenze.
Perciò la generalizzazione viene gestita tramite relazioni dedicate:
\texttt{IS\_LINEA}, \texttt{IS\_CARGO}, \texttt{IS\_PRIV}, \texttt{IS\_MIL},
ciascuna referenziante la chiave primaria dell’entità \texttt{Aereo}.  
Questa soluzione ci consente di associare un aereo a più categorie allo stesso tempo,
mantenendo un modello relazionale compatto e rispettando i requisiti funzionali.

\paragraph{Personale}
La generalizzazione su \textbf{Personale} è \emph{parziale e disgiunta}.  
Non tutte le persone sono piloti, controllori o manutentori, e un membro del personale
non può appartenere a più di una di queste categorie specializzate.  
La rimozione avviene introducendo tre relazioni:
\texttt{IS\_PILOTA}, \texttt{IS\_CTRL}, \texttt{IS\_MAN}.  
Poichè si tratta di una generalizzazione parziale, l'entità padre \texttt{Personale} non può essere eliminata,
visto che alcune persone non appartengono a nessun ruolo specializzato.

\paragraph{Log di Transito}
La generalizzazione su \textbf{LogTransito} è \emph{totale e disgiunta}.  
Ogni rilevazione di log deve rientrare necessariamente in una sola delle categorie stabilite.
La generalizzazione viene rimossa introducendo tre relazioni figlie:
\texttt{IS\_GPS}, \texttt{IS\_DATIVOLO}, \texttt{IS\_ERRORE}.  
L’entità padre \texttt{LogTransito} resta, poiché include gli attributi generali
(altitudine, velocità, timestamp) e svolge il ruolo di contenitore dei dati comuni.

\medskip
\noindent
Il diagramma E-R ristrutturato tiene conto di queste modifiche.


\begin{figure}[H] 
    \centering
    \includegraphics[width=\textwidth]{schema_er_ristrutturato.drawio.png}
    \caption{Diagramma Entità-Relazione del progetto STAR}
    \label{fig:schema_er}
\end{figure}


Lo schema ristrutturato contiene esclusivamente costrutti direttamente 
mappabili nel modello relazionale.  
Di seguito viene riportato lo schema logico completo, dove l’asterisco (*)
denota gli attributi che possono assumere valori nulli.

\begin{itemize}

\item \textbf{Aeroporto}(\underline{codice}, nome, citta, nazione, coordinate, numero\_piste, numero\_voli\_giornalieri)

\item \textbf{HubInternazionale}(\underline{codice}, numero\_terminal)  
\hspace{3mm} codice $\rightarrow$ Aeroporto(codice)

\item \textbf{Regionale}(\underline{codice}, lunghezza\_max\_pista, notturno)  
\hspace{3mm} codice $\rightarrow$ Aeroporto(codice)

\item \textbf{BaseMilitare}(\underline{codice}, codice\_nato, livello\_sicurezza)  
\hspace{3mm} codice $\rightarrow$ Aeroporto(codice)

\item \textbf{Gate}(\underline{codice}, \underline{numero\_gate}, piano, tunnel)  
\hspace{3mm} codice $\rightarrow$ Aeroporto(codice)

\item \textbf{Aereo}(\underline{matricola}, data\_immatricolazione, ore\_volo\_accumulate)

\item \textbf{AereoDiLinea}(\underline{matricola}, wifi*, intrattenimento*)  
\hspace{3mm} matricola $\rightarrow$ Aereo(matricola)

\item \textbf{AereoCargo}(\underline{matricola}, volume\_carico\_max)  
\hspace{3mm} matricola $\rightarrow$ Aereo(matricola)

\item \textbf{AereoPrivato}(\underline{matricola}, proprietario, servizi\_inclusi*)  
\hspace{3mm} matricola $\rightarrow$ Aereo(matricola)

\item \textbf{AereoMilitare}(\underline{matricola}, codice\_missione, ente\_operativo)  
\hspace{3mm} matricola $\rightarrow$ Aereo(matricola)

\item \textbf{TipoAereo}(\underline{id\_tipo}, costruttore, autonomia, velocita\_crociera)

\item \textbf{Componente}(\underline{id\_tipo}, \underline{numero\_parte}, descrizione)  
\hspace{3mm} id\_tipo $\rightarrow$ TipoAereo(id\_tipo)

\item \textbf{Volo}(\underline{id\_volo}, codice\_volo, data\_ora\_programmata, data\_ora\_effettiva*, stato, peso\_totale\_carico, matricola\_aereo) \\
\hspace{3mm} matricola\_aereo $\rightarrow$ Aereo(matricola)

\item \textbf{Partenza}(id\_volo, aeroporto\_origine)  
\hspace{3mm} id\_volo $\rightarrow$ Volo(id\_volo)  
\hspace{3mm} aeroporto\_origine $\rightarrow$ Aeroporto(codice)

\item \textbf{Arrivo}(id\_volo, aeroporto\_destinazione)  
\hspace{3mm} id\_volo $\rightarrow$ Volo(id\_volo)  
\hspace{3mm} aeroporto\_destinazione $\rightarrow$ Aeroporto(codice)

\item \textbf{FasciaOraria}(\underline{id\_slot}, ora\_inizio, ora\_fine, tipologia, stato\_slot)

\item \textbf{PrenotazioneDecollo}(id\_volo, id\_slot)  
\hspace{3mm} id\_volo $\rightarrow$ Volo(id\_volo)  
\hspace{3mm} id\_slot $\rightarrow$ FasciaOraria(id\_slot)

\item \textbf{PrenotazioneAtterraggio}(id\_volo, id\_slot)  
\hspace{3mm} id\_volo $\rightarrow$ Volo(id\_volo)  
\hspace{3mm} id\_slot $\rightarrow$ FasciaOraria(id\_slot)

\item \textbf{RottaPianificata}(\underline{id\_rotta}, nome, distanza, durata\_stimata, consumo\_previsto)

\item \textbf{AssegnazioneRotta}(id\_volo, id\_rotta)  
\hspace{3mm} id\_volo $\rightarrow$ Volo(id\_volo)  
\hspace{3mm} id\_rotta $\rightarrow$ RottaPianificata(id\_rotta)

\item \textbf{Personale}(\underline{id\_personale}, nome, cognome, data\_nascita, data\_assunzione)

\item \textbf{Pilota}(\underline{id\_personale}, numero\_licenza, ore\_volo, ultima\_visita)  
\hspace{3mm} id\_personale $\rightarrow$ Personale(id\_personale)

\item \textbf{ControlloreVolo}(\underline{id\_personale}, livello\_certificazione, scadenza\_abilitazione)  
\hspace{3mm} id\_personale $\rightarrow$ Personale(id\_personale)

\item \textbf{Manutentore}(\underline{id\_personale}, qualifica)  
\hspace{3mm} id\_personale $\rightarrow$ Personale(id\_personale)

\item \textbf{Equipaggio}(id\_personale, id\_volo, ruolo, ore\_servizio) \\
\hspace{3mm} id\_personale $\rightarrow$ Personale(id\_personale) \\
\hspace{3mm} id\_volo $\rightarrow$ Volo(id\_volo)

\item \textbf{Monitoraggio}(id\_personale, id\_volo, fase\_volo*) \\
\hspace{3mm} id\_personale $\rightarrow$ ControlloreVolo(id\_personale) \\
\hspace{3mm} id\_volo $\rightarrow$ Volo(id\_volo)

\item \textbf{Presidio}(id\_personale, codice, data\_inizio, data\_fine*, settore) \\
\hspace{3mm} id\_personale $\rightarrow$ ControlloreVolo(id\_personale) \\
\hspace{3mm} codice $\rightarrow$ Aeroporto(codice)

\item \textbf{Container}(\underline{codice\_container}, peso\_tara, tipologia)

\item \textbf{ImbarcoMerce}(id\_volo, codice\_container, peso\_registrato, data\_imbarco) \\
\hspace{3mm} id\_volo $\rightarrow$ Volo(id\_volo) \\
\hspace{3mm} codice\_container $\rightarrow$ Container(codice\_container)

\item \textbf{Trasporto}(matricola, codice\_container, data\_carico, posizione\_stiva) \\
\hspace{3mm} matricola $\rightarrow$ AereoCargo(matricola) \\
\hspace{3mm} codice\_container $\rightarrow$ Container(codice\_container)

\item \textbf{ClassePasseggeri}(\underline{id\_classe}, spazio\_gambe)

\item \textbf{Configurazione}(matricola, id\_classe, numero\_posti) \\
\hspace{3mm} matricola $\rightarrow$ AereoDiLinea(matricola) \\
\hspace{3mm} id\_classe $\rightarrow$ ClassePasseggeri(id\_classe)

\item \textbf{ServizioAggiuntivo}(\underline{id\_classe}, \underline{nome\_servizio}, descrizione)  
\hspace{3mm} id\_classe $\rightarrow$ ClassePasseggeri(id\_classe)

\item \textbf{LogTransito}(\underline{id\_log}, id\_volo, timestamp, altitudine*, velocita*) \\
\hspace{3mm} id\_volo $\rightarrow$ Volo(id\_volo)

\item \textbf{LogGPS}(\underline{id\_log}, latitudine, longitudine) \\
\hspace{3mm} id\_log $\rightarrow$ LogTransito(id\_log)

\item \textbf{LogDatiVolo}(\underline{id\_log}, livello\_carburante, pressione, temperatura) \\
\hspace{3mm} id\_log $\rightarrow$ LogTransito(id\_log)

\item \textbf{LogErrori}(\underline{id\_log}, codice\_errore, messaggio, severita) \\
\hspace{3mm} id\_log $\rightarrow$ LogTransito(id\_log)

\item \textbf{GestioneAlert}(id\_personale, id\_log, stato, data\_presa\_carico) \\
\hspace{3mm} id\_personale $\rightarrow$ Manutentore(id\_personale) \\
\hspace{3mm} id\_log $\rightarrow$ LogErrori(id\_log)

\end{itemize}

\end{document}